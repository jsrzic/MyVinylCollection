\chapter{Implementacija i korisničko sučelje}
		
		
		\section{Korištene tehnologije i alati}
		
%			\textbf{\textit{dio 2. revizije}}
%			
%			 \textit{Detaljno navesti sve tehnologije i alate koji su primijenjeni pri izradi dokumentacije i aplikacije. Ukratko ih opisati, te navesti njihovo značenje i mjesto primjene. Za svaki navedeni alat i tehnologiju je potrebno \textbf{navesti internet poveznicu} gdje se mogu preuzeti ili više saznati o njima}.
			
			Tehnologije koje se bile korištene u svrhu komunikacije tima su bile \underline{WhatsApp}\footnote{\url{https://www.whatsapp.com/}} i \underline{Discord}\footnote{\url{https://discord.com/}}. Za izradu obrazaca uporabe i UML dijagrama korišten je alat \underline{Astah UML}\footnote{\url{https://astah.net/}}, a kao sustav za upravljanje kodom projekta \underline{Git}\footnote{\url{https://git-scm.com/}}. Kompletni kod i dokumentacija projekta dostupna je na udaljenom repozitoriju platforme \underline{Gitlab}\footnote{\url{https://gitlab.com/}}.
			
			Kao razvojno okruženje korišteno je \underline{IntelliJ IDEA}\footnote{\url{https://www.jetbrains.com/idea/}}, integrirano razvojno okruženje razvijeno od tvrtke JetBrains. Napisano je u Javi u svrhu razvoja računalnog softvera. IDE pruža integraciju s alatima za izradu/pakiranje kao što su bower, gradle, i sl. Podržava sustave kao Git i SVN te baze podataka tipa Microsoft SQL Server, Oracle, PostrgeSQL, i sl.
			
			\textit{Frontend} dio web aplikacije napisan je korištenjem Javascript biblioteke \underline{React}\footnote{\url{https://reactjs.org/}} i programskog jezika \underline{JavaScript}\footnote{\url{https://www.javascript.com/}}, dok je \textit{backend} dio napisan je korištenjem radnog okvira \underline{Spring Boot}\footnote{\url{https://spring.io/projects/spring-boot}} i programskog jezika \underline{Java}\footnote{\url{https://www.java.com/en/}}. React (također poznat kao React.js ili ReactJS) open-source biblioteka za izgradnju korisničkih sučelja temeljenih na UI komponentama. React se bavi samo upravljanjem stanjem i prikazivanjem tog stanja DOM-u, s obzirom na to stvaranje aplikacija obično zahtijeva korištenje dodatnih biblioteka za interakciju s API-jem. Radni okvir Spring Boot olakšava stvaranje samostalnih aplikacija koje se temelje na Springu koje se mogu samo "pokrenuti".
			
			Za upravljanje relacijskom bazom podataka korišten je sustav \underline{PostgreSQL}\footnote{\url{https://www.postgresql.org/}}, a za pokretanje i rad aplikacije korištena je platforma \underline{Heroku}\footnote{\url{https://id.heroku.com/login}}. Dokumentacija je napisana u Latex uredniku \underline{Texmaker}\footnote{\url{https://www.xm1math.net/texmaker/}}, UML dijagrami su napravljeni u programu \underline{AstahUML}\footnote{\url{https://astah.net/}} i ispitivanje programskog rješenja je odrađeno korištenjem metode JUnit u Javi za ispitivanje komponenti i programa \underline{Selenium IDE}\footnote{\url{https://www.selenium.dev/selenium-ide/}} za ispitivanje sustava.

	
		\section{Ispitivanje programskog rješenja}
			
%			\textbf{\textit{dio 2. revizije}}\\
%			
%			 \textit{U ovom poglavlju je potrebno opisati provedbu ispitivanja implementiranih funkcionalnosti na razini komponenti i na razini cijelog sustava s prikazom odabranih ispitnih slučajeva. Studenti trebaju ispitati temeljnu funkcionalnost i rubne uvjete.}

		Za ispitivanje sustava koristio se Selenium IDE.
	
			
			\subsection{Ispitivanje komponenti}
%			\textit{Potrebno je provesti ispitivanje jedinica (engl. unit testing) nad razredima koji implementiraju temeljne funkcionalnosti. Razraditi \textbf{minimalno 6 ispitnih slučajeva} u kojima će se ispitati redovni slučajevi, rubni uvjeti te izazivanje pogreške (engl. exception throwing). Poželjno je stvoriti i ispitni slučaj koji koristi funkcionalnosti koje nisu implementirane. Potrebno je priložiti izvorni kôd svih ispitnih slučajeva te prikaz rezultata izvođenja ispita u razvojnom okruženju (prolaz/pad ispita). }
			
			
			
			\subsection{Ispitivanje sustava}
			
%			 \textit{Potrebno je provesti i opisati ispitivanje sustava koristeći radni okvir Selenium\footnote{\url{https://www.seleniumhq.org/}}. Razraditi \textbf{minimalno 4 ispitna slučaja} u kojima će se ispitati redovni slučajevi, rubni uvjeti te poziv funkcionalnosti koja nije implementirana/izaziva pogrešku kako bi se vidjelo na koji način sustav reagira kada nešto nije u potpunosti ostvareno. Ispitni slučaj se treba sastojati od ulaza (npr. korisničko ime i lozinka), očekivanog izlaza ili rezultata, koraka ispitivanja i dobivenog izlaza ili rezultata.\\ }
%			 
%			 \textit{Izradu ispitnih slučajeva pomoću radnog okvira Selenium moguće je provesti pomoću jednog od sljedeća dva alata:}
%			 \begin{itemize}
%			 	\item \textit{dodatak za preglednik \textbf{Selenium IDE} - snimanje korisnikovih akcija radi automatskog ponavljanja ispita	}
%			 	\item \textit{\textbf{Selenium WebDriver} - podrška za pisanje ispita u jezicima Java, C\#, PHP koristeći posebno programsko sučelje.}
%			 \end{itemize}
%		 	\textit{Detalji o korištenju alata Selenium bit će prikazani na posebnom predavanju tijekom semestra.}
%			

\textbf{Ispitni slučaj 1: Prijava u sustav - uspješna prijava}
			
				\noindent\textbf{Ulaz:}
				
					\begin{packed_enum}
					
						\item Otvaranje naslovne stranice web aplikacije
						\item Pomak kursora iznad polja za upis korisničkog imena
						\item Pritisak na polje za upis korisničkog imena
						\item Upis korisničkog imena
						\item Pomak kursora iznad polja za upis lozinke
						\item Pritisak na polje za upis lozinke
						\item Upis lozinke
						\item Pomak kursora iznad gumba za prijavu
						\item Pritisak gumba "Prijava"
						
					\end{packed_enum}
				
				\noindent\textbf{Očekivani rezultat:}
				
					\begin{packed_enum}
					
						\item Prikazuje se početna stranica web aplikacije
						
					\end{packed_enum}
				
				\noindent\textbf{Rezultat:} Očekivani rezultat je zadovoljen. \textcolor{green}{Aplikacija je prošla test.}
				
				\begin{figure}[H]
					\includegraphics[width=\textwidth]{SeleniumIDE_testovi/login_steps.png}
					\centering
					\caption{Potrebni koraci za uspješnu prijavu}
					\label{fig:login_steps}
				\end{figure}
				
				\begin{figure}[H]
					\includegraphics[width=\textwidth]{SeleniumIDE_testovi/login_home_page.png}
					\centering
					\caption{Početna stranica nakon uspješne prijave}
					\label{fig:login_home_page}
				\end{figure}

\noindent\textbf{Ispitni slučaj 2: Prijava u sustav - neuspješna prijava}
				
				\noindent\textbf{Ulaz:}
				
					\begin{packed_enum}
					
						\item Otvaranje naslovne stranice web aplikacije
						\item Pomak kursora iznad polja za upis korisničkog imena
						\item Pritisak na polje za upis korisničkog imena
						\item Upis pogrešnog korisničkog imena
						\item Pomak kursora iznad polja za upis lozinke
						\item Pritisak na polje za upis lozinke
						\item Upis lozinke
						\item Pomak kursora iznad gumba za prijavu
						\item Pritisak gumba "Prijava"
						
					\end{packed_enum}
				
				\noindent\textbf{Očekivani rezultat:}
				
					\begin{packed_enum}
					
						\item Prikazuje se naslovna stranica web aplikacije
						\item Sustav izabucje poruku o pogrešno upisanim korisničkim imenom ili lozinkom
						
					\end{packed_enum}
				
				\noindent\textbf{Rezultat:} Očekivani rezultat je zadovoljen. \textcolor{green}{Aplikacija je prošla test.}
				
				\begin{figure}[H]
					\includegraphics[width=\textwidth]{SeleniumIDE_testovi/login_fail_steps.png}
					\centering
					\caption{Potrebni koraci za neuspješnu prijavu}
					\label{fig:login_steps}
				\end{figure}
				
				\begin{figure}[H]
					\includegraphics[width=\textwidth]{SeleniumIDE_testovi/login_fail_frontpage.png}
					\centering
					\caption{Naslovna stranica nakon neuspješne prijave}
					\label{fig:login_home_page}
				\end{figure}
				
				\noindent\textbf{Ispitni slučaj 3: Dodavanje oglasa}
				\noindent\textbf{Ispitni slučaj 4: Stvaranje kolekcije}
			
			\eject 
		
		
		\section{Dijagram razmještaja}
			
%			\textbf{\textit{dio 2. revizije}}
%			
%			 \textit{Potrebno je umetnuti \textbf{specifikacijski} dijagram razmještaja i opisati ga. Moguće je umjesto specifikacijskog dijagrama razmještaja umetnuti dijagram razmještaja instanci, pod uvjetom da taj dijagram bolje opisuje neki važniji dio sustava.}

		Dijagram razmještaja je strukturni dijagram koji opisuje topologiju sustava i usredotočen je na odnos sklopovskih i programskih dijelova. Sustav je baziran na trorazinskoj arhitekturi pri kojoj klijent - poslužitelj i poslužitelj - baza podataka komuniciraju preko HTTP veze. Klijenti upotrebljavaju web preglednik kako bi pristupili web aplikaciji, dok web aplikacija pristupa bazi podataka kako bi pristupala podacima u svrhu provjere ili prikaza korisniku.

				\begin{figure}[H]
					\includegraphics[width=\textwidth]{slike/DD_web_aplikacija.png}
					\centering
					\caption{Dijagram razmještaja}
					\label{fig:dig_razmjestaja}
				\end{figure}
			
			\eject 
		
		\section{Upute za puštanje u pogon}
		
%			\textbf{\textit{dio 2. revizije}}\\
%		
%			 \textit{U ovom poglavlju potrebno je dati upute za puštanje u pogon (engl. deployment) ostvarene aplikacije. Na primjer, za web aplikacije, opisati postupak kojim se od izvornog kôda dolazi do potpuno postavljene baze podataka i poslužitelja koji odgovara na upite korisnika. Za mobilnu aplikaciju, postupak kojim se aplikacija izgradi, te postavi na neku od trgovina. Za stolnu (engl. desktop) aplikaciju, postupak kojim se aplikacija instalira na računalo. Ukoliko mobilne i stolne aplikacije komuniciraju s poslužiteljem i/ili bazom podataka, opisati i postupak njihovog postavljanja. Pri izradi uputa preporučuje se \textbf{naglasiti korake instalacije uporabom natuknica} te koristiti što je više moguće \textbf{slike ekrana} (engl. screenshots) kako bi upute bile jasne i jednostavne za slijediti.}
%			
%			
%			 \textit{Dovršenu aplikaciju potrebno je pokrenuti na javno dostupnom poslužitelju. Studentima se preporuča korištenje neke od sljedećih besplatnih usluga: \href{https://aws.amazon.com/}{Amazon AWS}, \href{https://azure.microsoft.com/en-us/}{Microsoft Azure} ili \href{https://www.heroku.com/}{Heroku}. Mobilne aplikacije trebaju biti objavljene na F-Droid, Google Play ili Amazon App trgovini.}
			
			
			
			\eject 
