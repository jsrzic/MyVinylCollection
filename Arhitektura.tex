\chapter{Arhitektura i dizajn sustava}
		
%		\textbf{\textit{dio 1. revizije}}\\
%
%		\textit{ Potrebno je opisati stil arhitekture te identificirati: podsustave, preslikavanje na radnu platformu, spremišta podataka, mrežne protokole, globalni upravljački tok i sklopovsko-programske zahtjeve. Po točkama razraditi i popratiti odgovarajućim skicama:}
%	\begin{itemize}
%		\item 	\textit{izbor arhitekture temeljem principa oblikovanja pokazanih na predavanjima (objasniti zašto ste baš odabrali takvu arhitekturu)}
%		\item 	\textit{organizaciju sustava s najviše razine apstrakcije (npr. klijent-poslužitelj, baza podataka, datotečni sustav, grafičko sučelje)}
%		\item 	\textit{organizaciju aplikacije (npr. slojevi frontend i backend, MVC arhitektura) }		
%	\end{itemize}

	Arhitektura sustava koju smo koristili se dijeli na tri podsustava:
	
		\begin{packed_item}
			\item Web poslužitelj
			\item Web aplikacija
			\item Baza podataka
		\end{packed_item}

		\begin{figure}[H]
			\includegraphics[width=\textwidth]{slike/arhitektura.png}
			\centering
			\caption{Arhitektura sustava}
			\label{fig:arhitektura}
		\end{figure}
		
		\textit{\underline{Web preglednik}} je aplikacija koja omogućuje korisniku pristup \textit{World Wide Web}-u. Kada korisnik zatraži web stranicu, preglednik dohvaća potreban sadržaj, web stranicu i potrebne multimedijske sadržaje, s web poslužitelja i prikazuje na korisnikovom uređaju. Web stranica je hipertekstualni dokument koji se prikazuje korisniku na ekranu. Web stranice se sastoje od jedne ili više tekstualnih datoteka pisanih u jezicima: HTML \textit{(HyperText Markup Lagnuage)}, CSS \textit{(Cascade Style Sheets)} te JS \textit{(JavaScript)}, koje web preglednik prevodi i prikazuje ih korisniku koji putem web preglednika šalje zahtjev web poslužitelju.
		
		\textit{\underline{Web poslužitelj}} je računalni hardverski sofver koji služi za komunikaciju klijenta s aplikacijom. Poslužitelj prima zahtjeve preko HTTP \textit{(Hyper Text Transfer Protocol)} mrežnog protokola stvorenog za distirbuciju web sadržaja. Korisnik preko web preglednika pokreće komunikaciju postavljanjem željenih zahtjeva koristeći HTTP, a web poslužitelj, ovisno o zahtjevu koji obrađuje, odgovara sadržajem tog resursa (HTML dokumentom) ili porukom o pogrešci.
		
		Prilikom izrade projekta korišteni su sustav PostgreSQL (za upravljanje relacijskom bazom podataka), programski jezik Java i razvojni okvir Spring Boot (backend) te programski jezik JavaScript i razvojni okvir React (frontend). Odabrano razvojno okruženje je IntelliJ te cloud platforma Heroku (za deploy aplikacije). Arhitektura sustava temelji se na MVC \textit{(Model-View-Controller)} obrascu koji funkcionira na načelu dizajna SoC \textit{(Separation of Concerns)}. SoC omogućava odvajanje programa u različite odjeljke tako da se svaki odjeljak bavi svojim dijelom, što olakšava izradu i daljnji razvoj aplikacije.
		
		MVC sastoji se od:
		\begin{packed_item}
			\item \textbf{Model} - središnja komponenta sustava koja izravno upravlja podacima, logikom i pravilima aplikacije. Neovisna je o korisničkom sučelju te prima ulazne podatke od Controllera.
			
			\item \textbf{View} - definira kako se podaci aplikacije trebaju prikazivati
			
			\item \textbf{Controller} - zadužen je za primanje i uređivanje zahtjeva koje prosljeđuje na Model ili View. Upravalja korisničkim zahtjevima i na temelju njih izvodi daljnje korake potrebne s ostalim elementima.
		\end{packed_item}
		
		\eject
		

				
		\section{Baza podataka}
			
%			\textbf{\textit{dio 1. revizije}}\\
%			
%		\textit{Potrebno je opisati koju vrstu i implementaciju baze podataka ste odabrali, glavne komponente od kojih se sastoji i slično.}

\quad Za potrebe sustava našeg projekta korištena je relacijska baza podataka. Osnovna građa baze je relacija, odnosno tablica koja je definirana imenom i skupom atributa. Glavni zadatak baze podataka je brza i jednostavna pohrana, izmjena i dohvat podataka za obradu. Baza podataka ove aplikacije sastoji se od tablica:
			
			\begin{packed_item}
				\item User
				\item Genre
				\item Subgenre
				\item Vinyl
				\item Artist
				\item Ad
				\item Sale ad
				\item Exchange ad
				\item Event
			\end{packed_item}
		
			\subsection{Opis tablica}
			
%
%				\textit{Svaku tablicu je potrebno opisati po zadanom predlošku. Lijevo se nalazi točno ime varijable u bazi podataka, u sredini se nalazi tip podataka, a desno se nalazi opis varijable. Svjetlozelenom bojom označite primarni ključ. Svjetlo plavom označite strani ključ}
				
\quad\textbf{User}\quad Ovaj entitet sadrži sve važne informacije o korisniku aplikacije. Sadrži atribute: user\_ id, name, surname, username, password, email, contact\_ email i location. Ovaj entitet je u vezi \textit{Many-to-Many} sa samim sobom preko atributa user\_ id korisnika, \textit{Many-to-Many} sa entitetom Genre preko genre\_ id, \textit{One-to-Many} sa entitetom Ad preko user\_ id, \textit{Many-to-Many} sa Vinyl preko user\_ id te \textit{One-to-Many} sa Vinyl preko vinyl\_ id i \textit{Many-to-Many} sa Artist preko user\_ id.
				
				%User 
				\begin{longtblr}[
					label=none,
					entry=none
					]{
						width = \textwidth,
						colspec={|X[8,l]|X[6, l]|X[20, l]|}, 
						rowhead = 1,
					} %definicija širine tablice, širine stupaca, poravnanje i broja redaka naslova tablice
					\hline \multicolumn{3}{|c|}{\textbf{User}}	 \\ \hline[3pt]
					\SetCell{LightGreen}user\_ id & LONG	&  Jedinstveni identifikator korisnika 	\\ \hline
					name	& STRING & Ime korisnika  	\\ \hline 
					surname & STRING & Prezime korisnika  \\ \hline 
					username & STRING	&  Korisničko ime	\\ \hline 
					password & STRING & Lozinka \\ \hline
					email & STRING & E-mail korisnika \\ \hline
					contact\_ email & STRING & E-mail za stupanje u kontakt s korisnikom \\ \hline
					location & GEOGRAPHY & Geografska lokacija uređaja korisnika \\ \hline
				\end{longtblr}
				
\textbf{Genre}\quad Ovaj entitet sadrži sve važne informacije o žanrovima. Sadrži atribute: genre\_ id i name. Ovaj entitet je u vezi \textit{Many-to-Many} sa User preko atributa user\_ id, \textit{One-to-Many} sa Subgenre preko genre\_ id i \textit{One-to-Many} sa Vinyl preko genre\_ id.
				
				%Genre
				\begin{longtblr}[
					label=none,
					entry=none
					]{
						width = \textwidth,
						colspec={|X[8,l]|X[6, l]|X[20, l]|}, 
						rowhead = 1,
					} 
					\hline \multicolumn{3}{|c|}{\textbf{Genre}}	 \\ \hline[3pt]
					\SetCell{LightGreen}genre\_ id & LONG	&  Jedinstveni identifikator žanra  \\ \hline
					name	 & STRING & Ime žanra  	\\ \hline 
				\end{longtblr}
				
\textbf{Subgenre}\quad	Ovaj entitet sadrži sve važne informacije o podžanrovima. Sadrži atribute: subgenre\_ id, name i parent\_ genre\_ id. Ovaj entitet je u vezi \textit{Many-to-One} sa Genre preko atributa genre\_ id.
				
				%Subgenre
				\begin{longtblr}[
					label=none,
					entry=none
					]{
						width = \textwidth,
						colspec={|X[8,l]|X[6, l]|X[20, l]|}, 
						rowhead = 1,
					} 
					\hline \multicolumn{3}{|c|}{\textbf{Subgenre}}	 \\ \hline[3pt]
					\SetCell{LightGreen}subgenre\_ id & LONG	&  Jedinstveni identifikator podžanra  \\ \hline
					name	& STRING & Ime podžanra  	\\ \hline
					\SetCell{LightBlue}parent\_ genre\_ id & LONG & Identifikator roditeljskog žanra \\ \hline
					
				\end{longtblr}
				
\textbf{Vinyl}\quad	Ovaj entitet sadrži sve važne informacije o vinilima. Sadrži atribute: vinyl\_ id, album\_ name, release\_ year, genre\_ id, subgenre\_ id, image, condition\_ evaluation, isRare, description, value\_ kn, RPM, diameter, capacity, reproduction\_ quality, nmb\_ of\_ audio\_ channels, time\_ of\_ reproduction. Ovaj entitet je u vezi \textit{One-to-Many} sa Ad preko atributa vinyl\_ id, \textit{Many-to-Many} sa User preko user\_ id, \textit{Many-to-One} sa User preko vinyl\_ id, \textit{Many-to-One} sa Genre preko parent\_ genre\_ id i \textit{Many-to-One} sa Artist preko artist\_ id.
				
				%Vinyl
				\begin{longtblr}[
					label=none,
					entry=none
					]{
						width = \textwidth,
						colspec={|X[8,l]|X[6, l]|X[20, l]|}, 
						rowhead = 1,
					} 
					\hline \multicolumn{3}{|c|}{\textbf{Vinyl}}	 \\ \hline[3pt]
					\SetCell{LightGreen}vinyl\_ id & LONG	&  Jedinstveni identifikator vinila  \\ \hline
					album\_ name	& STRING & Ime albuma \\ \hline 
					release\_ year & INT & Godina izdanja \\ \hline
					\SetCell{LightBlue}genre\_ id & LONG & Jedinstveni identifikator žanra \\ \hline
					subgenre\_ id & LONG & Jednistveni identifikator podžanra \\ \hline
					image & BYTEA & Slika vinila \\ \hline
					condition\_ evaluation & INT & Ocjena stanja vinila \\ \hline
					isRare & BOOLEAN & Raritet vinila \\ \hline
					description & STRING & Kratki opis vinila \\ \hline
					value\_ kn & NUMERIC & Vrijednost vinila \\ \hline
					RPM & STRING & Brzina okretaja u minuti \\ \hline
					diameter & FLOAT & Promjer vinila \\ \hline
					capacity & STRING & Kapacitet vinila \\ \hline
					reproduction\_ quality & STRING & Kvaliteta reprodukcije vinila \\ \hline
					nmb\_ of\_ audio\_ channels & INT & Broj audio kanala \\ \hline
					time\_ of\_ reproduction & TIME & Vrijeme reprodukcije \\ \hline
					
				\end{longtblr}
				
\textbf{Artist}\quad	Ovaj entitet sadrži sve važne informacije o izvođaču. Sadrži atribute: artist\_ id i name. Ovaj entitet je u vezi \textit{Many-to-Many} sa User preko atributa user\_ id i \textit{One-to-Many} sa Vinyl preko artist\_ id.
				
				%Artist
				\begin{longtblr}[
					label=none,
					entry=none
					]{
						width = \textwidth,
						colspec={|X[8,l]|X[6, l]|X[20, l]|}, 
						rowhead = 1,
					} 
					\hline \multicolumn{3}{|c|}{\textbf{Artist}}	 \\ \hline[3pt]
					\SetCell{LightGreen}artist\_ id & LONG	&  Jedinstveni identifikator izvođača  \\ \hline
					name & STRING & Ime izvođača \\ \hline
					
				\end{longtblr}
				
\textbf{Ad}\quad	Ovaj entitet sadrži sve važne informacije o oglasu. Sadrži atribute: ad\_ id, vinyl\_ id, sale\_ ad\_ id, exchange\_ ad\_ id i isActive. Ovaj entitet je u vezi \textit{Many-to-One} sa User preko user\_ id, \textit{Many-to-One} sa Vinyl preko vinyl\_ id, \textit{One-to-One} sa SaleAd preko ad\_ id i \textit{One-to-One} sa ExchangeAd preko ad\_ id.
				
				%Ad
				\begin{longtblr}[
					label=none,
					entry=none
					]{
						width = \textwidth,
						colspec={|X[8,l]|X[6, l]|X[20, l]|}, 
						rowhead = 1,
					} 
					\hline \multicolumn{3}{|c|}{\textbf{Ad}}	 \\ \hline[3pt]
					\SetCell{LightGreen}ad\_ id & LONG	&  Jedinstveni identifikator oglasa  \\ \hline
					vinyl\_ id & STRING & Jedinstveni identifikator vinila \\ \hline
					\SetCell{LightBlue}sale\_ ad\_ id & LONG & Jednistveni identifikator oglasa za prodaju \\ \hline
					\SetCell{LightBlue}exchange\_ ad\_ id & LONG & Jedinstveni identifikator oglasa za razmjenu \\ \hline
					isActive & BOOLEAN & Aktivnost oglasa \\ \hline
					
				\end{longtblr}
				
\textbf{SaleAd}\quad	Ovaj entitet sadrži sve važe informacije za oglas koji je označen za prodaju vinila. Sadrži atribute: ad\_ id i price. Ovaj entitet je u vezi \textit{One-to-One} sa Ad preko atributa ad\_ id.
				
				%SaleAd
				\begin{longtblr}[
					label=none,
					entry=none
					]{
						width = \textwidth,
						colspec={|X[8,l]|X[6, l]|X[20, l]|}, 
						rowhead = 1,
					} 
					\hline \multicolumn{3}{|c|}{\textbf{SaleAd}}	 \\ \hline[3pt]
					\SetCell{LightGreen}ad\_ id & LONG	&  Jedinstveni identifikator oglasa  \\ \hline
					price & NUMERIC & Cijena vinila u oglasu \\ \hline
					
				\end{longtblr}
				
\textbf{ExchangeAd}\quad	Ovaj entitet sadrži sve važe informacije za oglas koji je označen za razmjenu vinila. Sadrži atribute: ad\_ id, new\_ Owner\_ id i exchanged\_ vinyl\_ id. Ovaj entitet je u vezi \textit{One-to-One} sa Ad preko atributa ad\_ id.
				
				%ExchangeAd
				\begin{longtblr}[
					label=none,
					entry=none
					]{
						width = \textwidth,
						colspec={|X[8,l]|X[6, l]|X[20, l]|}, 
						rowhead = 1,
					} 
					\hline \multicolumn{3}{|c|}{\textbf{ExchangeAd}}	 \\ \hline[3pt]
					\SetCell{LightGreen}ad\_ id & LONG	&  Jedinstveni identifikator oglasa  \\ \hline
					new\_ Owner\_ id & LONG & Jedinstveni identifikator novog vlasnika \\ \hline
					exchanged\_ vinyl\_ id & LONG & Jedinstveni identifikator razmjenjenog vinila \\ \hline
					
				\end{longtblr}
				
\textbf{Event}\quad	Ovaj entitet sadrži sve važe informacije za događaj. Sadrži atribute: event\_ id, title, image, creation\_ date, description, social\_ network\_ link i location. Ovaj entitet nije u vezi niti sa jednim drugim entitetom.
				
				%Event
				\begin{longtblr}[
					label=none,
					entry=none
					]{
						width = \textwidth,
						colspec={|X[8,l]|X[6, l]|X[20, l]|}, 
						rowhead = 1,
					} 
					\hline \multicolumn{3}{|c|}{\textbf{Event}}	 \\ \hline[3pt]
					\SetCell{LightGreen}event\_ id & LONG	&  Jedinstveni identifikator događaja  \\ \hline
					title & STRING & Naslov događaja \\ \hline
					image & BYTEA & Slika događaja \\ \hline
					creation\_ date & DATE & Datum događaja \\ \hline
					description & STRING & Kratki opis događaja \\ \hline
					social\_ network\_ link & STRING & Link na događaj \\ \hline
					location & GEOGRAPHY & Lokacija događaja \\ \hline
					
				\end{longtblr}
			
			\subsection{Dijagram baze podataka}
%				\textit{ U ovom potpoglavlju potrebno je umetnuti dijagram baze podataka. Primarni i strani ključevi moraju biti označeni, a tablice povezane. Bazu podataka je potrebno normalizirati. Podsjetite se kolegija "Baze podataka".}

				\begin{figure}[H]
					\includegraphics[width=\textwidth]{slike/baza.jpg}
					\centering
					\caption{Dijagram baze podataka}
					\label{fig:baza}
				\end{figure}
			
			
			\eject
			
			
		\section{Dijagram razreda}
		
%			\textit{Potrebno je priložiti dijagram razreda s pripadajućim opisom. Zbog preglednosti je moguće dijagram razlomiti na više njih, ali moraju biti grupirani prema sličnim razinama apstrakcije i srodnim funkcionalnostima.}\\
%			
%			\textbf{\textit{dio 1. revizije}}\\
%			
%			\textit{Prilikom prve predaje projekta, potrebno je priložiti potpuno razrađen dijagram razreda vezan uz \textbf{generičku funkcionalnost} sustava. Ostale funkcionalnosti trebaju biti idejno razrađene u dijagramu sa sljedećim komponentama: nazivi razreda, nazivi metoda i vrste pristupa metodama (npr. javni, zaštićeni), nazivi atributa razreda, veze i odnosi između razreda.}\\

\quad Slike \ref{fig:controller} i \ref{fig:model} prikazuju dijagrame razreda koji pripadaju \textit{backend} dijelu MVC arhitekture. Razredi prikazan na slici \ref{fig:controller} nasljeđuju Controller razred. Neki metode implementirane u tim razredima manipuliraju s DTO \textit{(Data transfer object)} koji su dohvaćeni pomoću metoda implementiranih u Model razredima. Metode implementirane u Controller razredima vraćaju JSON datoteke s HTML status kodom. Razredi su podijeljeni logički po pravu pristupa metodama određenih aktora.

				\begin{figure}[H]
					\includegraphics[width=\textwidth]{slike/CD_controller.png}
					\centering
					\caption{Dijagram razreda controller}
					\label{fig:controller}
				\end{figure}
				
			\eject
				
			Na slici \ref{fig:model} prikazan je dijagram razreda modela koji služi za preslikavanje baze podataka u aplikaciji. Implementirane metode direktno komuniciraju s bazom podataka te vraćaju tražene podatke. Razred User predstavlja prijavljenog korisnika aplikacije te ima sve navedene atribute koji prijvaljeni korisnik mora imati kako bi mogao komunicirati s bazom podataka. Razred Ad predstavlja oglase koje korisnik može objaviti a razred Vinyl predstavlja vinile koje korisnik posjeduje i vinil koji se nalazi u oglasu.
				
				\begin{figure}[H]
					\includegraphics[width=\textwidth]{slike/CD_model.png}
					\centering
					\caption{Dijagram razreda model}
					\label{fig:model}
				\end{figure}

			
%			\textbf{\textit{dio 2. revizije}}\\			
%			
%			\textit{Prilikom druge predaje projekta dijagram razreda i opisi moraju odgovarati stvarnom stanju implementacije}
%			
			
			
			\eject
		
		\section{Dijagram stanja}
			
			
%			\textbf{\textit{dio 2. revizije}}\\
%			
%			\textit{Potrebno je priložiti dijagram stanja i opisati ga. Dovoljan je jedan dijagram stanja koji prikazuje \textbf{značajan dio funkcionalnosti} sustava. Na primjer, stanja korisničkog sučelja i tijek korištenja neke ključne funkcionalnosti jesu značajan dio sustava, a registracija i prijava nisu. }

		Dijagram stanja opisuje dinamičko ponašanje dijela sustava te prikazuje stanja objekta i prijelaze iz jednog stanja u drugo temeljene na događajima. Na \ref{fig:dig_stanja} prikazan je dijagram stanja za registriranog korisnika. Nakon prijave, korisniku se prikazuje početna stranica na kojoj može pristupiti svom profilu, oglasima, kolekcijama te svojim prijateljima. Na svom korisničkom profilu može mijenjati podatke ili izbrisati račun u slučaju prestanka korištenja aplikacije. Korisnici imaju mogućnost stvaranja kolekcija vinila svojih preferenci te oglašavanja vinila za razmjenu ili prodaju. Prilikom razmjene ili prodaje vinila kada korisnici stupaju u kontakt, imaju mogućnost praćenja, tj. dodavanja kao prijatelja, korisnika.

				\begin{figure}[H]
					\includegraphics[width=\textwidth]{slike/STDM_korisnik.png}
					\centering
					\caption{Dijagram stanja}
					\label{fig:dig_stanja}
				\end{figure}
			
			
			\eject 
		
		\section{Dijagram aktivnosti}
			
%			\textbf{\textit{dio 2. revizije}}\\
%			
%			 \textit{Potrebno je priložiti dijagram aktivnosti s pripadajućim opisom. Dijagram aktivnosti treba prikazivati značajan dio sustava.}

			Dijagram aktivnosti primjenjuju se za modeliranje poslovnih procesa te upravljačkog i podaktovnog toka. Na \ref{fig:dig_aktivnosti} prikazuje se tijek stvaranja oglasa od početne do završne točke pokazujući različite smjerove aktivnosti koje se izvršavaju. Radnja završava kada su sve radnje, potrebne za stvaranje oglasa, izpravno izvedene.

				\begin{figure}[H]
					\includegraphics[width=\textwidth]{slike/AD_stvaranje_oglasa.png}
					\centering
					\caption{Dijagram aktivnosti}
					\label{fig:dig_aktivnosti}
				\end{figure}
			
			\eject
		\section{Dijagram komponenti}
		
%			\textbf{\textit{dio 2. revizije}}\\
%		
%			 \textit{Potrebno je priložiti dijagram komponenti s pripadajućim opisom. Dijagram komponenti treba prikazivati strukturu cijele aplikacije.}

			Dijagram komponenti na \ref{fig:dig_komponenti} prikazuje povezanost i međuovisnost komponenti, interne strukture i odnos prema vanjskoj okolini. Web aplikacija temeljena je na oblikovnom obrascu MVC (\textit{Model-View-Controller}). REST API radi s podacima koji pripadaju \textit{backend} dijelu aplikacije. SQL API omogućuje interakciju sa tablicama i podacima iz baze podataka pomoću SQL upita. Podaci koji se dohvaćaju iz baze podataka proslijeđuju se MVC strukturi u obliku DTO (\textit{Data Transfer Object}).

				\begin{figure}[H]
					\includegraphics[width=\textwidth]{slike/CMPD_web_aplikacija.png}
					\centering
					\caption{Dijagram komponenti}
					\label{fig:dig_komponenti}
				\end{figure}
				
			\eject

			