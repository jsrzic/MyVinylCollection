\chapter{Zaključak i budući rad}
		
%		\textbf{\textit{dio 2. revizije}}\\
%		
%		 \textit{U ovom poglavlju potrebno je napisati osvrt na vrijeme izrade projektnog zadatka, koji su tehnički izazovi prepoznati, jesu li riješeni ili kako bi mogli biti riješeni, koja su znanja stečena pri izradi projekta, koja bi znanja bila posebno potrebna za brže i kvalitetnije ostvarenje projekta i koje bi bile perspektive za nastavak rada u projektnoj grupi.}
%		
%		 \textit{Potrebno je točno popisati funkcionalnosti koje nisu implementirane u ostvarenoj aplikaciji.}

		Projektni zadatak naše grupe bio je izrada i razvoj web aplikacije za stvaranje kolekcije vinila uz mogućnost razmjene, kupovine i prodaje vinila. Zadatak je ostvaren nakon 16 tjedana grupnog rada tima. Provedba praćenja napretka razvoja aplikacije tima bila je kroz dvije faze.
		
		Prva faza projekta krenula je sporije zbog upoznavanja tima i njihovih iskustva i znanja, organizacije i raspodijele poslova te dogovoru za sastanke koji će pratiti napredak projekta. Uspješnom raspodijelom tima na podtimove obilježio je početak rada na projetku. Cilj prve faze projekta bila je izrada \textit{alfa} inačice projekta, tj. web aplikacija s osnovim radnim funkcionalnostima kao što su \textit{login} i \textit{registracija}. Velika pomoć u izradi i razvoju \textit{frontend-a} i \textit{backend-a} bili su obrasci i dijagrami (obrasci uporabe, sekvencijski dijagrami, dijgram razreda i model baze podataka). Vizualni dizajn web aplikacije bio je brzo riješen kako bi tim mogao nastaviti dalje implementirati funkcije.
		
		Druga faza projekta, iako kraća od prve, ali zahtijevnija, krenula je intenzivnije po pitanju posla koji je trebao svaki član tima odraditi. Prilikom izrade implementacijskih rješenja članovi tima morali su samostalno naučiti odabrane alate i programske jezike kako bi zadovoljili dogovoreni cilj. Također je bilo potrebno napraviti korektne UML dijagrame i ispravnu dokumentaciju kako bi budući korisnici, investitori mogli lakše razumijeti funckionalnost sustava. Kvalitetno izrađen prototip aplikacije i složnost tima, pomogao nam je u drugoj fazi projekta kako bi se mogle izbjeći vremensko složne pogreške i nepotrebne funckionalnosti aplikacije. Slijedeći korak projektnog zadatka bio bi izrada mobline aplikacije koja bi mogla otvoriti nove funckionalnosti aplikacije (npr. prepoznavanje ploče s kamerom koja bi popunila osnovne informacije ploče, ...).
		
		Najvažniji dio projekta bila je organizacija i komunikacija tima. Prateći napredak i ostatak posla svakog podtima bila je nužna kako bi se projektni zadatak mogao napraviti u roku. Svaki član tima sudjelovanjem u ovom projektu shvatio je veliku važnost u dobroj vremenskoj organiziaciji i komunikaciji i koordinaciji s ostatkom tima. Zadovoljni smo s količinom posla koje smo odradili u zadanom vremenskom periodu i postoji još mjesta za poboljšanje, ali to je sve posljedica neiskustva, koje svi u početku imamo, na takvim i sličnim budućim projektima. 
		
		\eject 