\chapter{Opis projektnog zadatka}
		
%		\textbf{\textit{dio 1. revizije}}\\
%		
%		\textit{Na osnovi projektnog zadatka detaljno opisati korisničke zahtjeve. Što jasnije opisati cilj projektnog zadatka, razraditi problematiku zadatka, dodati nove aspekte problema i potencijalnih rješenja. Očekuje se minimalno 3, a poželjno 4-5 stranica opisa.	Teme koje treba dodatno razraditi u ovom poglavlju su:}
%		\begin{packed_item}
%			\item \textit{potencijalna korist ovog projekta}
%			\item \textit{postojeća slična rješenja (istražiti i ukratko opisati razlike u odnosu na zadani zadatak). Dodajte slike koja predočavaju slična rješenja.}
%			\item \textit{skup korisnika koji bi mogao biti zainteresiran za ostvareno rješenje.}
%			\item \textit{mogućnost prilagodbe rješenja }
%			\item \textit{opseg projektnog zadatka}
%			\item \textit{moguće nadogradnje projektnog zadatka}
%		\end{packed_item}
%		
%		\textit{Za pomoć pogledati reference navedene u poglavlju „Popis literature“, a po potrebi konzultirati sadržaj na internetu koji nudi dobre smjernice u tom pogledu.}
%		\eject
	
		\text Cilj ovog projekta je stvaranje web aplikacije \textit{"My Vinyl Collection"} koja će služiti kao mjesto na kojem će korisnici, koji su zainteresirani, ljubitelji ili kolekcionari vinila, pronalaziti nove vinile, prodavati, kupovati ili razmjenjivati vinile s drugima. U planu nam je olakšati korisnicima traženje tako da korištenjem njihove lokacije prikazujemo oglase u njihovoj blizini.\\
		
		\textit{\underline{Neregistrirani korisnici}} mogu samo pregledavati i pretraživati tuđe kolekcije i korisnike. Neregistriranom korisniku je omogućeno prijavljivanje postojećim računom (upisivanjem korisničkog imena i lozinke) ili kreiranjem novog računa. Za kreiranje novog računa potrebni su: 
		\begin{packed_item}
			\item ime
			\item prezime
			\item korisničko ime
			\item e-mail adresa
			\item lozinka
			\item potvrda lozinke
			\item odabir preferiranih žanrova
			\item lokacija uređaja
		\end{packed_item}
Registracijom u sustav korisniku se dodijeljuju određene mogućnosti i značajke.\\

	 \textit{\underline{Registrirani korisnici}} imaju mogućnosti pregledavanja, mijenjanja osobnih podataka i brisanja profila. Također mogu izrađivati kolekcije i podkolekcije vinila koje posjeduju, objavljivati oglase za prodaju ili razmjenu vinila te imaju pregled kupljenih, aktivnih i prodanih oglasa, tj. vinila. Prilikom objavljivanja oglasa, ako korisnik nema odabranu lokaciju na profilu, od njega se traži odabir lokacije vinila. S obzirom na kojoj su korisnici lokaciji, koju će moći promijeniti ručno u postavkama, na početnoj stranici će im se prikazivati svi vinili žanrova koje je korisnik odabrao pri registraciji. Na početnoj stranici će također korisnici imat mogućnost pretraživanja drugih korisnika, pretraživanja i filtriranja oglasa i kolekcija, mogućnost praćenja drugih korisnika te označavanja njihovih vinila sa "Sviđa mi se". Prilikom dodavanja vinila u kolekcije i podkolekcije korisnici će biti obvezni napisati naziv izvođača, naziv albuma, godinu izdanja, žanr vinila te osnovne parametre o samom fizičkom vinilu kao što su: promjer, RPM (\textit{revolutions per minute}), kapacitet, kvaliteta reprodukcije i broj audio kanala. Vezano uz glavne karakteristike vinila korisnici mogu opcionalno napisati podžanr, ocjenu stanja, sliku, te kao dodatak raritet, opis i vrijednost vinila u kunama. Svaka vrijednost vinila će automatski biti preračunata i prikazana, ispod cijene u kunama, u eurima. Kad je registrirani korisnik zainteresiran za oglas može napisati poruku vlasniku oglasa kojem se poruka šalje na mail s kojim se registrirao. E-mail adresa za primanje poruka i sličnih obavijesti se može promijeniti u postavkama te može se razlikovati od registracijskog e-maila.\\
	 
	  \text Tijekom razmjene ili kupovine, tj. prodaje kad vinil promijeni vlasnika, ne nalazi se više u kolekciji prodavatelja. Na profilu korisnika postoji lista svih vinila koje je korisnik prodao ili zamijenio.\\
	 
	 Uloga \textit{\underline{administratora}} je potrebna za vođenje računa o objavljenim oglasima i o tome kakav je objavljeni sadržaj. Administratori imaju ovlasti brisanja oglasa, blokiranja i aktiviranja korisničkog računa. Također su u mogućnosti kreiranja događaja iz vanjskih usluga, kao što su \textit{Facebook, Instagram, Twitter} i slično, o raznim događanjima vezanim za ovu temu (npr. buvljaci, slušaone i slično). Tijekom kreiranja svakog događaja registriranim korisnicima će biti ponuđene osnovne informacije kao što su naslov, slike, datum kratki opis te link na događaj.\\
	 
	 
	 \section {Primjeri sličnih rješenja}
	 
	 \begin{packed_enum}
	 		\item Njuškalo
	 		
	 		\text Njuškalo je web stranica na kojoj se objavljaju raznovrsni oglasi od automobila i drugih motornih vozila,  nekretnina, pa sve do hrane i pića i popusta u raznim trgovinama. Služi kao mjesto za trgovinu ne samo za korisnike nego i za tvrtke, obrtnike, udruge te agencije, dok je naš projekt web stranica za ljubitelje, kolekcionare i prodavače vinila.\\
	 		
	 		\begin{figure}[H]
	 			\includegraphics[width = \linewidth]{slike/njuskalo.png}
	 			\centering
	 			\caption{Početna stranica Njuškala}
	 			\label{fig:njuškalo}
	 		\end{figure}
	 		
	 		
		 	\item Oglasnik
	 		
	 		\text Oglasnik je slična web stranica kao i Njuškalo koja ima malo drugačijih značajki za razliku od naših. Od mogućnosti procjene vrijednosti auta, kalkulatora uvoza vozila, online plaćanja pa do korisničke podrške, pravila o postupanju kolačića, uvjeta korištenja i uvjeta kupnje. Također postoji opcija prijave oglasa u slučaju da krši pravila oglašavanja, mogućnost pronalaska ostalih oglasa u blizini (koji se na karti prikažu ovisno o tome koliko se umanji/uveća karta). \\
	 		
	 		\begin{figure}[H]
	 			\includegraphics[width = \linewidth]{slike/oglasnik.png}
	 			\centering
	 			\caption{Početna stranica Oglasnika}
	 			\label{fig:oglasnik}
	 		\end{figure}
	 		
	 	\end{packed_enum}
	 		
	 		\text Sljedeći korak bi mogao biti proširenje na oglašavanje, tj. prodaju, kupovinu ili razmjenu gramofona te suradnja s tvrtkama, obrtnicima i ostalima za promoviranje njihovih proizvoda i usluga.
	 		