%definira klasu dokumenta 
\documentclass[12pt]{report} 

%prostor izmedu naredbi \documentclass i \begin{document} se zove uvod. U njemu se nalaze naredbe koje se odnose na cijeli dokument

%osnovni LaTex ne može riješiti sve probleme, pa se koriste različiti paketi koji olakšavaju izradu željenog dokumenta
\usepackage[croatian]{babel} 
\usepackage{amssymb}
\usepackage{amsmath}
\usepackage{txfonts}
\usepackage{mathdots}
\usepackage{titlesec}
\usepackage{array}
\usepackage{lastpage}
\usepackage{etoolbox}
\usepackage{tabularray}
\usepackage{color, colortbl}
\usepackage{adjustbox}
\usepackage{geometry}
\usepackage[classicReIm]{kpfonts}
\usepackage{hyperref}
\usepackage{fancyhdr}
\usepackage{graphicx}

\usepackage{float}
\usepackage{setspace}
\restylefloat{table}


\patchcmd{\chapter}{\thispagestyle{plain}}{\thispagestyle{fancy}}{}{} %redefiniranje stila stranice u paketu fancyhdr

%oblik naslova poglavlja
\titleformat{\chapter}{\normalfont\huge\bfseries}{\thechapter.}{20pt}{\Huge}
\titlespacing{\chapter}{0pt}{0pt}{40pt}


\linespread{1.3} %razmak između redaka

\geometry{a4paper, left=1in, top=1in,}  %oblik stranice

\hypersetup{ colorlinks, citecolor=black, filecolor=black, linkcolor=black,	urlcolor=black }   %izgled poveznice


%prored smanjen između redaka u nabrajanjima i popisima
\newenvironment{packed_enum}{
	\begin{enumerate}
		\setlength{\itemsep}{0pt}
		\setlength{\parskip}{0pt}
		\setlength{\parsep}{0pt}
	}{\end{enumerate}}

\newenvironment{packed_item}{
	\begin{itemize}
		\setlength{\itemsep}{0pt}
		\setlength{\parskip}{0pt}
		\setlength{\parsep}{0pt}
	}{\end{itemize}}




%boja za privatni i udaljeni kljuc u tablicama
\definecolor{LightBlue}{rgb}{0.9,0.9,1}
\definecolor{LightGreen}{rgb}{0.9,1,0.9}

%Promjena teksta za dugačke tablice
\DefTblrTemplate{contfoot-text}{normal}{Nastavljeno na idućoj stranici}
\SetTblrTemplate{contfoot-text}{normal}
\DefTblrTemplate{conthead-text}{normal}{(Nastavljeno)}
\SetTblrTemplate{conthead-text}{normal}
\DefTblrTemplate{middlehead,lasthead}{normal}{Nastavljeno od prethodne stranice}
\SetTblrTemplate{middlehead,lasthead}{normal}

%podesavanje zaglavlja i podnožja

\pagestyle{fancy}
\lhead{Programsko inženjerstvo}
\rhead{My Vinyl Collection}
\lfoot{sCROZ prosječni}
\cfoot{stranica \thepage/\pageref{LastPage}}
\rfoot{\today}
\renewcommand{\headrulewidth}{0.2pt}
\renewcommand{\footrulewidth}{0.2pt}


\begin{document} 
	
	
	
	\begin{titlepage}
		\begin{center}
			\vspace*{\stretch{1.0}} %u kombinaciji s ostalim \vspace naredbama definira razmak između redaka teksta
			\LARGE Programsko inženjerstvo\\
			\large Ak. god. 2021./2022.\\
			
			\vspace*{\stretch{3.0}}
			
			\huge My Vinyl Collection\\
			\Large Dokumentacija, Rev. \textit{0.9}\\
			
			\vspace*{\stretch{12.0}}
			\normalsize
			Grupa: \textit{sCROZ prosječni}\\
			Voditelj: \textit{Josip Srzić}\\
			
			
			\vspace*{\stretch{1.0}}
			Datum predaje: \textit{18. 11. 2021.}\\
	
			\vspace*{\stretch{4.0}}
			
			Nastavnik: \textit {Vlado Sruk}\\
		
		\end{center}

	
	\end{titlepage}

	
	\tableofcontents


	\chapter{Dnevnik promjena dokumentacije}
		
%		\textbf{\textit{Kontinuirano osvježavanje}}\\
				
		
		\begin{longtblr}[
				label=none
			]{
				width = \textwidth, 
				colspec={|X[2]|X[13]|X[3]|X[3]|}, 
				rowhead = 1
			}
			\hline
			\textbf{Rev.}	& \textbf{Opis promjene/dodatka} & \textbf{Autori} & \textbf{Datum}\\[3pt] \hline
			0.1 & Napravljen predložak.	& Toni Drezga & 29.10.2021. 		\\[3pt] \hline 
			0.2	& Dopisan dodatak i dnevnik promjene dokumentacije.  & Toni Drezga & 31.10.2021. 	\\[3pt] \hline 
			0.3  & Napisan dio Opisa projektnog zadatka. & Toni Drezga & 9.11.2021. \\[3pt] \hline
			0.4 & Napisan cijeli Opis projektnog zadatka i ažurirani sastanci. & Toni Drezga & 12.11.2021. \\[3pt] \hline
			0.5 & Napisani UC-ovi. & Toni Drezga & 13.11.2021. \\[3pt] \hline 
			0.6 & Dodani grafovi UC-ova i napisani tekstovi sekvencijskih dijagrama  & Toni Drezga & 15.11.2021. \\[3pt] \hline
			0.7 & Opis arhitekture i dizajna sustava & Toni Drezga & 16.11.2021. \\[3pt] \hline  
			0.8 & Završeni UC-ovi i sekvencijski dijagrami. Gotovo 3. poglavlje. & Toni Drezga & 17.11.2021. \\[3pt] \hline 
			0.9 & Izmjene u dokumentaciji: dodani neki UC-ovi, neki UC-ovi prepravljeni, sekvencijski dijagrami prepravljeni & Josip Srzić, Tea Krišto, Marta Dulibić, Toni Drezga & 18.11.2021. \\[3pt] \hline 
			0.10 & Sekvencijski dijagrami & * & 09.09.2013. \\[3pt] \hline 
			\textbf{1.0} & Verzija samo s bitnim dijelovima za 1. ciklus & * & 11.09.2013. \\[3pt] \hline 
			
%			1.1 & Uređivanje teksta -- funkcionalni i nefunkcionalni zahtjevi & * \newline * & 14.09.2013. \\[3pt] \hline 
%			1.2 & Manje izmjene:Timer - Brojilo vremena & * & 15.09.2013. \\[3pt] \hline 
%			1.3 & Popravljeni dijagrami obrazaca uporabe & * & 15.09.2013. \\[3pt] \hline 
%			1.5 & Generalna revizija strukture dokumenta & * & 19.09.2013. \\[3pt] \hline 
%			1.5.1 & Manja revizija (dijagram razmještaja) & * & 20.09.2013. \\[3pt] \hline 
%			\textbf{2.0} & Konačni tekst predloška dokumentacije  & * & 28.09.2013. \\[3pt] \hline 
%			&  &  & \\[3pt] \hline	
		\end{longtblr}
	
	\chapter{Opis projektnog zadatka}
		
%		\textbf{\textit{dio 1. revizije}}\\
%		
%		\textit{Na osnovi projektnog zadatka detaljno opisati korisničke zahtjeve. Što jasnije opisati cilj projektnog zadatka, razraditi problematiku zadatka, dodati nove aspekte problema i potencijalnih rješenja. Očekuje se minimalno 3, a poželjno 4-5 stranica opisa.	Teme koje treba dodatno razraditi u ovom poglavlju su:}
%		\begin{packed_item}
%			\item \textit{potencijalna korist ovog projekta}
%			\item \textit{postojeća slična rješenja (istražiti i ukratko opisati razlike u odnosu na zadani zadatak). Dodajte slike koja predočavaju slična rješenja.}
%			\item \textit{skup korisnika koji bi mogao biti zainteresiran za ostvareno rješenje.}
%			\item \textit{mogućnost prilagodbe rješenja }
%			\item \textit{opseg projektnog zadatka}
%			\item \textit{moguće nadogradnje projektnog zadatka}
%		\end{packed_item}
%		
%		\textit{Za pomoć pogledati reference navedene u poglavlju „Popis literature“, a po potrebi konzultirati sadržaj na internetu koji nudi dobre smjernice u tom pogledu.}
%		\eject
	
		\text Cilj ovog projekta je stvaranje web aplikacije \textit{"My Vinyl Collection"} koja će služiti kao mjesto na kojem će se korisnici, koji su zainteresirani, ljubitelji ili kolekcionari, registrirati kako bi pronalazili nove vinile, prodavali, kupovali ili razmjenjivali vinile s drugima.
		Prilikom pokretanja aplikacije korisnicima će automatski biti ponuđene kolekcije i vinili iz Zagreba te će moći pretraživati tuđe kolekcije i podkolekcije.\\
		
		\textit{\underline{Neregistrirani korisnici}} mogu samo pregledavati i pretraživati tuđe kolekcije i korisnike. Prilikom učitavanja aplikacije, korisniku će se prikazivati, automatski, oglasi iz Zagreba. Neregistriranom korisniku je omogućeno prijavljivanje postojećim računom (upisivanjem korisničkog imena i lozinke) ili kreiranjem novog računa. Za kreiranje novog računa potrebni su: 
		\begin{packed_item}
			\item ime
			\item prezime
			\item korisničko ime
			\item e-mail adresa
			\item lozinka
			\item potvrda lozinke
			\item odabir preferiranih žanrova
			\item lokacija uređaja
		\end{packed_item}
Registracijom u sustav korisniku se dodijeljuju određene mogućnosti i značajke.\\

	 \textit{\underline{Registrirani korisnici}} imaju mogućnosti pregledavanja, mijenjanja osobnih podataka i brisanja profila. Također mogu izrađivati kolekcije i podkolekcije vinila koje posjeduju, objavljivati oglase za prodaju ili razmjenu vinila te imaju pregled kupljenih, aktivnih i prodanih oglasa, tj. vinila. Prilikom objavljivanja oglasa, ako korisnik nema odabranu lokaciju na profilu, od njega se traži odabir lokacije vinila. S obzirom na kojoj su korisnici lokaciji, koju će moći promijeniti ručno u postavkama, na početnoj stranici će im se prikazivati svi vinili koji su fizički u njihovoj blizini. Na početnoj stranici će također korisnici imat mogućnost pretraživanja drugih korisnika te pretraživanja i filtriranja vinila. Prilikom dodavanja vinila u kolekcije i podkolekcije korisnici će biti obvezni napisati naziv izvođača, naziv albuma, godinu izdanja, žanr vinila te osnovne parametre o samom fizičkom vinilu kao što su: promjer, RPM (\textit{revolutions per minute}), kapacitet, kvaliteta reprodukcije i broj audio kanala. Vezano uz glavne karakteristike vinila korisnici mogu opcionalno napisati podžanr, ocjenu stanja [Poor (P)/Fair (F), Good (G), Very Good (VG), Very Good Plus (VG+), Excellent (E), Near Mint (NM), Mint (M)], sliku, te kao dodatak raritet, opis i vrijednost vinila u kunama. Svaka vrijednost vinila će automatski biti preračunata i prikazana, ispod cijene u kunama, u eurima. Kad je registrirani korisnik zainteresiran za oglas može napisati poruku vlasniku oglasa kojem se poruka šalje na mail s kojim se registrirao. E-mail adresa za primanje poruka i sličnih obavijesti se može promijeniti u postavkama te može se razlikovati od registracijskog e-maila.\\
	 
	  \text Tijekom razmjene ili kupovine, tj. prodaje kad vinil promijeni vlasnika, ne nalazi se više u kolekciji prodavatelja. Na profilu korisnika postoji lista svih vinila koje je korisnik prodao ili zamijenio. Vinili se također mogu prodavati na "crno", tj. prodavatelj može ručno označiti vinil kao prodan da se više ne prikazuje na stranici.\\
	 
	 Uloga \textit{\underline{administratora}} je potrebna za vođenje računa o objavljenim oglasima i o tome kakav je objavljeni sadržaj. Administratori imaju ovlasti brisanja oglasa, blokiranja i aktiviranja korisničkog računa. Također su u mogućnosti kreiranja događaja iz vanjskih usluga, kao što su \textit{Facebook, Instagram, Twitter} i slično, o raznim događanjima vezanim za ovu temu (npr. buvljaci, slušaone i slično). Tijekom kreiranja svakog događaja registriranim korisnicima će biti ponuđene osnovne informacije kao što su naslov, slike, datum kratki opis te link na događaj.\\
	 
	 
	 \section {Primjeri sličnih rješenja}
	 
	 \item[] \begin{packed_enum}
	 		\item Njuškalo
	 		
	 		\text Njuškalo je web stranica na kojoj se objavljaju raznovrsni oglasi od automobila i drugih motornih vozila,  nekretnina, pa sve do hrane i pića i popusta u raznim trgovinama. Služi kao mjesto za trgovinu ne samo za korisnike nego i za tvrtke, obrtnike, udruge te agencije, dok je naš projekt web stranica za ljubitelje, kolekcionare i prodavače vinila.\\
	 		
	 		\begin{figure}[H]
	 			\includegraphics[width = \linewidth]{slike/njuskalo.png}
	 			\centering
	 			\caption{Početna stranica Njuškala}
	 			\label{fig:njuškalo}
	 		\end{figure}
	 		
	 		
		 	\item Oglasnik
	 		
	 		\text Oglasnik je slična web stranica kao i Njuškalo koja ima malo drugačijih značajki za razliku od naših. Od mogućnosti procjene vrijednosti auta, kalkulatora uvoza vozila, online plaćanja pa do korisničke podrške, pravila o postupanju kolačića, uvjeta korištenja i uvjeta kupnje. Također postoji opcija prijave oglasa u slučaju da krši pravila oglašavanja, mogućnost pronalaska ostalih oglasa u blizini (koji se na karti prikažu ovisno o tome koliko se umanji/uveća karta). \\
	 		
	 		\begin{figure}[H]
	 			\includegraphics[width = \linewidth]{slike/oglasnik.png}
	 			\centering
	 			\caption{Početna stranica Oglasnika}
	 			\label{fig:oglasnik}
	 		\end{figure}
	 		
	 	\end{packed_enum}
	 		
	 		\text Sljedeći korak bi mogao biti proširenje na oglašavanje, tj. prodaju, kupovinu ili razmjenu gramofona te suradnja s tvrtkama, obrtnicima i ostalima za promoviranje njihovih proizvoda i usluga.
	 		
	\chapter{Specifikacija programske potpore}
		
	\section{Funkcionalni zahtjevi}
			
%			\textbf{\textit{dio 1. revizije}}\\
%			
%			\textit{Navesti \textbf{dionike} koji imaju \textbf{interes u ovom sustavu} ili  \textbf{su nositelji odgovornosti}. To su prije svega korisnici, ali i administratori sustava, naručitelji, razvojni tim.}\\
%				
%			\textit{Navesti \textbf{aktore} koji izravno \textbf{koriste} ili \textbf{komuniciraju sa sustavom}. Oni mogu imati inicijatorsku ulogu, tj. započinju određene procese u sustavu ili samo sudioničku ulogu, tj. obavljaju određeni posao. Za svakog aktora navesti funkcionalne zahtjeve koji se na njega odnose.}\\
			
			
			\noindent \textbf{Dionici:}
			
			\begin{packed_enum}
				
				\item Korisnici
				
					\begin{packed_enum}
					
						\item Neregistrirani
						\item Registrirani
						
					\end{packed_enum}
					
				\item Administrator			
				\item Razvojni tim
				
			\end{packed_enum}
			
			\noindent \textbf{Aktori i njihovi funkcionalni zahtjevi:}
			
			\begin{packed_enum}
			
				\item  \underbar{Neregistrirani/neprijavljeni (inicijator) može:}
				
				\begin{packed_enum}
				
					\item pregledavati i pretraživati tuđe kolekcije
					\item pregledavati i pretraživati korisnike
					\item registrirati se u sustav, stvoriti novi korisnički račun za koji su mu potrebni ime, prezime, korisničko ime, e-mail adresa, lozinka, odabir preferiranih žanrova i lokacija uređaja
					
				\end{packed_enum}
			
				\item  \underbar{Registrirani/prijavljeni (sudionik) može:}
				
				\begin{packed_enum}
				
					\item pregledavati i mijenjati osobne podatke
					
						\begin{packed_enum}
						
								\item promijeniti e-mail adresu za primanje poruka
								
						\end{packed_enum}
						
					\item izbrisati profil
					\item stvarati kolekcije i podkolekcije vinila
					\item objavljivati oglase za prodaju ili razmjenu
					\item pregledati kupljene, aktivne i prodane oglase (vinile)
					\item dodavati vinile u kolekcije i podkolekcije uz koje će biti obvezni napisati naziv izvođača, albuma, godinu izdanja te žanr vinila
					
					\begin{packed_enum}
					
						\item napisati osnovne parametre fizičkog vinila: promjer, RPM, kapacitet, kvaliteta reprodukcije, broj audio kanala
						\item opcionalno napisati glavne karakteristike vinila: podžanr, ocjena stanja, slika, raritet, opis, cijena 
						
					\end{packed_enum}
					
					\item poslati poruku korisniku za zainteresiranost u oglas
					\item označiti oglas (vinil) kao da je prodan da se više ne prikazuje na stranici
							
				\end{packed_enum}
				
				\item  \underbar{Administrator (inicijator) može:}
				
					\begin{packed_enum}
					
						\item vidjeti popis svih registriranih korisnika i njihovih podataka
						\item brisati oglase koji su u suprotnosti s pravilima korištenja aplikacije
						\item blokirati i aktivirati korisnički račun
						\item kreirati događaje iz vanjskih usluga (npr. Facebook, Instagram, Twitter i slično) o raznim događanjima (npr. buvljaci, slušaone i slično)
						
					\end{packed_enum}
					
				\item  \underbar{Baza podataka (inicijator):}
				
					\begin{packed_enum}
					
						\item pohranjuje sve podatke o korisnicima i njihovim ovlastima
						\item pohranjuje sve podatke o vinilima, njihovim karakteristikama i vlasnicima
						
					\end{packed_enum}	
									
				
			\end{packed_enum}
			
			\eject 
			
			
				
			\subsection{Obrasci uporabe}
				
%				\textbf{\textit{dio 1. revizije}}
				
				\subsubsection{Opis obrazaca uporabe}
%					\textit{Funkcionalne zahtjeve razraditi u obliku obrazaca uporabe. Svaki obrazac je potrebno razraditi prema donjem predlošku. Ukoliko u nekom koraku može doći do odstupanja, potrebno je to odstupanje opisati i po mogućnosti ponuditi rješenje kojim bi se tijek obrasca vratio na osnovni tijek.}\\
					

					\noindent \underbar{\textbf{UC1 - Pregled kolekcija i korisnika}}
					\begin{packed_item}
	
						\item \textbf{Glavni sudionik: }Neregistrirani, registrirani korisnik
						\item  \textbf{Cilj:} Pregledati tuđe vinile
						\item  \textbf{Sudionici:} Baza podataka
						\item  \textbf{Preduvjet:} -
						\item  \textbf{Opis osnovnog tijeka:}
						
						\item[] \begin{packed_enum}
	
							\item Automatski su prikazani oglasi, za neregistrirane korisnike, iz Zagreba prilikom učitavanja aplikacije
							\item Korisnik može pretraživati korisnike
							\item Korisnik može vidjeti kolekcije i podkolekcije pretraženog korisnika

						\end{packed_enum}
						
						\item  \textbf{Opis mogućih odstupanja:}
						
						\item[] \begin{packed_item}
	
							\item[1.a] Za registriranog korisnika se prikazuju oglasi ovisno o njegovoj lokaciji
							
						\end{packed_item}						
					\end{packed_item}
					
				\noindent \underbar{\textbf{UC2 - Registracija}}
					\begin{packed_item}
	
						\item \textbf{Glavni sudionik: }Neregistrirani korisnik
						\item  \textbf{Cilj:} Stvoriti korisnički račun za pristup sustavu
						\item  \textbf{Sudionici:} Baza podataka
						\item  \textbf{Preduvjet:} -
						\item  \textbf{Opis osnovnog tijeka:}
						
						\item[] \begin{packed_enum}
	
							\item Korisnik odabire opciju za registraciju
							\item Korisnik unosi potrebne korisničke podatke
							\item Korisnik prima obavijest o uspješnoj registraciji

						\end{packed_enum}
						
						\item  \textbf{Opis mogućih odstupanja:}
						
						\item[] \begin{packed_item}
	
							\item[2.a] Odabir već zauzetog korisničkog imena i/ili e-maila, unos korisničkih podataka u nedozvoljenom formatu ili pružanje neispravnih podataka
							
								\begin{packed_item}
								
									\item Sustav obavještava korisnika o neuspjeloj registraciji i vraća ga na stranicu registracije
									\item Korisnik mijenja podatke i završava registraciju ili odustaje od registracije
								
								\end{packed_item}
							
						\end{packed_item}					
					\end{packed_item}
					
				\noindent \underbar{\textbf{UC3 - Prijava}}
					\begin{packed_item}
	
						\item \textbf{Glavni sudionik: }Registrirani korisnik
						\item  \textbf{Cilj:} Dobiti pristup korinsičkom sučelju
						\item  \textbf{Sudionici:} Baza podataka
						\item  \textbf{Preduvjet:} Registracija
						\item  \textbf{Opis osnovnog tijeka:}
						
						\item[] \begin{packed_enum}
	
							\item Unos korisničkog imena i lozinke
							\item Potvrda o ispravnosti unesenih podataka
							\item Pristup korisničkim funkcijama
							
						\end{packed_enum}
						
						\item  \textbf{Opis mogućih odstupanja:}
						
						\item[] \begin{packed_item}
	
							\item[2.a] Neispravno korisničko ime/lozinka
							
								\begin{packed_item}
								
									\item Sustav obavještava korisnika o neuspjelom upisu
									
								\end{packed_item}
							
						\end{packed_item}				
					\end{packed_item}
					
				\noindent \underbar{\textbf{UC4 - Pregled osobnih podataka}}
					\begin{packed_item}
	
						\item \textbf{Glavni sudionik: }Registrirani korisnik
						\item  \textbf{Cilj:} Promjeniti osobne podatke
						\item  \textbf{Sudionici:} Baza podataka
						\item  \textbf{Preduvjet:} Korisnik je prijavljen
						\item  \textbf{Opis osnovnog tijeka:}
						
						\item[] \begin{packed_enum}
	
							\item Korsnik odabire opciju za pregled podataka
							\item Otvara se stranica s osobnim podacima korisnika

						\end{packed_enum}						
					\end{packed_item}
					
				\noindent \underbar{\textbf{UC5 - Promjena osobnih podataka}}
					\begin{packed_item}
	
						\item \textbf{Glavni sudionik: }Registrirani korisnik
						\item  \textbf{Cilj:} Promijeniti/ažurirati osobne podatke
						\item  \textbf{Sudionici:} Baza podataka
						\item  \textbf{Preduvjet:} Korisnik je prijavljen
						\item  \textbf{Opis osnovnog tijeka:}
						
						\item[] \begin{packed_enum}
	
							\item Korisnik odabire opciju za promjenu podataka
							\item Otvara se stranica s osobnim podacima korisnika
							\item Korisnik mijenja svoje osobne podatke
							\item Korisnik sprema promjene
							\item Baza podataka se ažurira

						\end{packed_enum}
						
						\item  \textbf{Opis mogućih odstupanja:}
						
						\item[] \begin{packed_item}
	
							\item[2.a] Korisnik promijeni svoje osobne podatke, ali ne odabere opciju "Spremi promjenu"
							
								\begin{packed_item}
									\item Sustav obavještava korisnika da nije spremio podatke prije izlaska iz prozora
								\end{packed_item}
							
						\end{packed_item}						
					\end{packed_item}
					
				\noindent \underbar{\textbf{UC6 - Brisanje korisničkog računa}}
					\begin{packed_item}
	
						\item \textbf{Glavni sudionik: }Registrirani korisnik
						\item  \textbf{Cilj:} Izbrisati svoj korisnički račun
						\item  \textbf{Sudionici:} Baza podataka
						\item  \textbf{Preduvjet:} Korisnik je prijavljen
						\item  \textbf{Opis osnovnog tijeka:}
						
						\item[] \begin{packed_enum}
	
							\item Korisnik odabire opciju za pregled podataka
							\item Otvara se stranica s osobnim podacima korisnika
							\item Korisnik sprema promjene
							\item Baza podataka se ažurira

						\end{packed_enum}
						
						\item  \textbf{Opis mogućih odstupanja:}
						
						\item[] \begin{packed_item}
	
							\item[2.a] Korisnik promijeni svoje osobne podatke, ali ne odabere opciju "Spremi promjenu"
							
								\begin{packed_item}
									\item Sustav obavještava korisnika da nije spremio podatke prije izlaska iz prozora
								\end{packed_item}
							
						\end{packed_item}						
					\end{packed_item}
					
				\noindent \underbar{\textbf{UC7 - Pregled aktivnih/neaktivnih oglasa}}
					\begin{packed_item}
	
						\item \textbf{Glavni sudionik: }Registrirani korisnik
						\item  \textbf{Cilj:} Pogledati trenutno aktivne/neaktivne oglase
						\item  \textbf{Sudionici:} Baza podataka
						\item  \textbf{Preduvjet:} Korisnik je prijavljen
						\item  \textbf{Opis osnovnog tijeka:}
						
						\item[] \begin{packed_enum}
	
							\item Korisnik odabire opciju za pregled oglasa
							\item Otvara se stranica sa svim oglasima
							\item Korisnik filtrira koje oglase hoće pregledati
							\item Aplikacija prikazuje filtrirane oglase

						\end{packed_enum}
						
						\item  \textbf{Opis mogućih odstupanja:}
						
						\item[] \begin{packed_item}
	
							\item[2.a] Korisnik nema aktivnih/neaktivih oglasa
							
						\end{packed_item}					
					\end{packed_item}
					
				\noindent \underbar{\textbf{UC8 - Dodavanje oglasa}}
					\begin{packed_item}
	
						\item \textbf{Glavni sudionik: }Registrirani korisnik
						\item  \textbf{Cilj:} Dodati novi oglas
						\item  \textbf{Sudionici:} Baza podataka
						\item  \textbf{Preduvjet:} Korisnik je prijavljen
						\item  \textbf{Opis osnovnog tijeka:}
						
						\item[] \begin{packed_enum}
	
							\item Korisnik odabire opciju za pregled oglasa
							\item Otvara se stranica sa svim oglasima
							\item Korisnik odabire opciju za dodavanje oglasa
							\item Korisnik odabire je li vinil u oglasu za prodaju ili zamjenu
							\item Promjene se upisuju u bazu podataka

						\end{packed_enum}						
					\end{packed_item}
					
				\noindent \underbar{\textbf{UC9 - Uređivanje oglasa}}
					\begin{packed_item}
	
						\item \textbf{Glavni sudionik: }Registrirani korisnik
						\item  \textbf{Cilj:} Urediti/ažurirati oglas
						\item  \textbf{Sudionici:} Baza podataka
						\item  \textbf{Preduvjet:} Korisnik je prijavljen
						\item  \textbf{Opis osnovnog tijeka:}
						
						\item[] \begin{packed_enum}
	
							\item Korisnik odabire opciju za pregled oglasa
							\item Otvara se stranica sa svim oglasima
							\item Korisnik odabire opciju uređivanja nad određenim oglasom
							\item Korisnik dodaje/izmjenjuje neke informacije u oglasu
							\item Nakon uređivanja sprema promjenu
							\item Promjene se upisuju u bazu podataka

						\end{packed_enum}						
					\end{packed_item}
					
				\noindent \underbar{\textbf{UC10 - Uklanjanje oglasa}}
					\begin{packed_item}
	
						\item \textbf{Glavni sudionik: }Registrirani korisnik, administrator
						\item  \textbf{Cilj:} Urediti/ažurirati oglas
						\item  \textbf{Sudionici:} Baza podataka
						\item  \textbf{Preduvjet:} Korisnik je prijavljen
						\item  \textbf{Opis osnovnog tijeka:}
						
						\item[] \begin{packed_enum}
	
							\item Korisnik odabire opciju za pregled oglasa
							\item Otvara se stranica sa svim oglasima
							\item Korisnik odabire opciju uklanjanja nad određenim oglasom
							\item Nakon uklanjanja sprema promjenu
							\item Promjene se upisuju u bazu podataka

						\end{packed_enum}					
					\end{packed_item}
					
				\noindent \underbar{\textbf{UC11 - Stvaranje kolekcije/podkolekcije}}
					\begin{packed_item}
	
						\item \textbf{Glavni sudionik: }Registrirani korisnik
						\item  \textbf{Cilj:} Grupirati vinile u novu kolekciju/podkolekciju
						\item  \textbf{Sudionici:} Baza podataka
						\item  \textbf{Preduvjet:} Korisnik je prijavljen
						\item  \textbf{Opis osnovnog tijeka:}
						
						\item[] \begin{packed_enum}
	
							\item Korisnik odabire opciju za pregled oglasa
							\item Otvara se stranica sa svim oglasima
							\item Korisnik odabire opciju stvaranja kolekcije/podkolekcije
							\item Korisnik dodaje vinile u kolekciju/podkolekciju
							\item Nakon dodavanja sprema promjenu
							\item Promjene se upisuju u bazu podataka

						\end{packed_enum}						
					\end{packed_item}
					
				\noindent \underbar{\textbf{UC12 - Uređivanje kolekcije/podkolekcije}}
					\begin{packed_item}
	
						\item \textbf{Glavni sudionik: }Registrirani korisnik
						\item  \textbf{Cilj:} Uređivati/ažurirati vinile u novu kolekciju/podkolekciju
						\item  \textbf{Sudionici:} Baza podataka
						\item  \textbf{Preduvjet:} Korisnik je prijavljen
						\item  \textbf{Opis osnovnog tijeka:}
						
						\item[] \begin{packed_enum}
	
							\item Korisnik odabire opciju za pregled oglasa
							\item Otvara se stranica sa svim oglasima i kolekcijama/podkolekcijama
							\item Korisnik odabire opciju uređivanja kolekcije/podkolekcije
							\item Korisnik dodaje/uklanja vinile u/iz kolekciju/podkolekciju
							\item Nakon ažuriranja sprema promjenu
							\item Promjene se upisuju u bazu podataka

						\end{packed_enum}						
					\end{packed_item}
					
				\noindent \underbar{\textbf{UC13 - Uklanjanje kolekcije/podkolekcije}}
					\begin{packed_item}
	
						\item \textbf{Glavni sudionik: }Registrirani korisnik
						\item  \textbf{Cilj:} Grupirati vinile u novu kolekciju/podkolekciju
						\item  \textbf{Sudionici:} Baza podataka
						\item  \textbf{Preduvjet:} Korisnik je prijavljen
						\item  \textbf{Opis osnovnog tijeka:}
						
						\item[] \begin{packed_enum}
	
							\item Korisnik odabire opciju za pregled oglasa
							\item Otvara se stranica sa svim oglasima i kolekcijama/podkolekcijama
							\item Korisnik odabire opciju brisanja kolekcije/podkolekcije
							\item Nakon dodavanja sprema promjenu
							\item Promjene se upisuju u bazu podataka

						\end{packed_enum}						
					\end{packed_item}
					
				\noindent \underbar{\textbf{UC14 - Javljanje na oglas}}
					\begin{packed_item}
	
						\item \textbf{Glavni sudionik: }Registrirani korisnik
						\item  \textbf{Cilj:} Javiti drugom korisniku za interesiranje u oglas
						\item  \textbf{Sudionici:} Baza podataka
						\item  \textbf{Preduvjet:} Korisnik je prijavljen
						\item  \textbf{Opis osnovnog tijeka:}
						
						\item[] \begin{packed_enum}
	
							\item Korisnik odabire korisnički profil iz oglasa
							\item Otvara se stranica odabranog korisnika
							\item Korisnik šalje e-mail

						\end{packed_enum}						
					\end{packed_item}
					
				\noindent \underbar{\textbf{UC15 - Pregled korisnika}}
					\begin{packed_item}
	
						\item \textbf{Glavni sudionik: }Administrator
						\item  \textbf{Cilj:} Pregledati registrirane korisnike
						\item  \textbf{Sudionici:} Baza podataka
						\item  \textbf{Preduvjet:} Korisnik je prijavljen i dodijeljena su mu prava administratora
						\item  \textbf{Opis osnovnog tijeka:}
						
						\item[] \begin{packed_enum}
	
							\item Administrator odabire opciju pregledavanja korisnika
							\item Prikaže se lista svih ispravno registriranih korisnika s osobnim podacima

						\end{packed_enum}						
					\end{packed_item}
					
				\noindent \underbar{\textbf{UC16 - Blokiranje korisnika}}
					\begin{packed_item}
	
						\item \textbf{Glavni sudionik: }Administrator
						\item  \textbf{Cilj:} Pregledati registrirane korisnike
						\item  \textbf{Sudionici:} Baza podataka
						\item  \textbf{Preduvjet:} Korisnik je prijavljen i dodijeljena su mu prava administratora
						\item  \textbf{Opis osnovnog tijeka:}
						
						\item[] \begin{packed_enum}
	
							\item Administrator odabire opciju blokiranja korisnika
							\item Administrator pronalazi željenog korisnika
							\item Administrator blokira željenog korisnika

						\end{packed_enum}
						
						\item  \textbf{Opis mogućih odstupanja:}
						
						\item[] \begin{packed_item}
	
							\item[3.a] Korisnik je već blokiran
							
								\begin{packed_item}
								
									\item Sustav javlja administratoru da je korisnik već blokiran
								
								\end{packed_item}
							
						\end{packed_item}						
					\end{packed_item}
					
				\noindent \underbar{\textbf{UC17 - Aktivacija korisnika}}
					\begin{packed_item}
	
						\item \textbf{Glavni sudionik: }Administrator
						\item  \textbf{Cilj:} Pregledati registrirane korisnike
						\item  \textbf{Sudionici:} Baza podataka
						\item  \textbf{Preduvjet:} Korisnik je prijavljen i dodijeljena su mu prava administratora
						\item  \textbf{Opis osnovnog tijeka:}
						
						\item[] \begin{packed_enum}
	
							\item Administrator odabire opciju aktivacije korisnika
							\item Administrator pronalazi željenog korisnika
							\item Administrator aktivira željenog korisnika

						\end{packed_enum}
						
						\item  \textbf{Opis mogućih odstupanja:}
						
						\item[] \begin{packed_item}
	
							\item[3.a] Korisnik je aktivan
							
								\begin{packed_item}
								
									\item Sustav javlja administratoru da je korisnik već aktiviran
								
								\end{packed_item}
							
						\end{packed_item}
					\end{packed_item}
					
					
				\noindent \underbar{\textbf{UC18 - Kreiranje događaja}}
					\begin{packed_item}
	
						\item \textbf{Glavni sudionik: }Administrator
						\item  \textbf{Cilj:} Kreirati događaj
						\item  \textbf{Sudionici:} Baza podataka
						\item  \textbf{Preduvjet:} Korisnik je prijavljen i dodijeljena su mu prava administratora
						\item  \textbf{Opis osnovnog tijeka:}
						
						\item[] \begin{packed_enum}
	
							\item Administrator odabire opciju stvaranja događaja
							\item Administrator upisuje datum i naziv događaja te stavlja neke slike, kratki opis i link na događaj
							\item Administrator objavljuje događaj na početnu stranicu
							\item Promjene se upisuju u bazu podataka

						\end{packed_enum}	
					\end{packed_item}
					
				
					
				\subsubsection{Dijagrami obrazaca uporabe}
					
					\textit{Prikazati odnos aktora i obrazaca uporabe odgovarajućim UML dijagramom. Nije nužno nacrtati sve na jednom dijagramu. Modelirati po razinama apstrakcije i skupovima srodnih funkcionalnosti.}
				\eject		
				
			\subsection{Sekvencijski dijagrami}
				
				\textbf{\textit{dio 1. revizije}}\\
				
				\textit{Nacrtati sekvencijske dijagrame koji modeliraju najvažnije dijelove sustava (max. 4 dijagrama). Ukoliko postoji nedoumica oko odabira, razjasniti s asistentom. Uz svaki dijagram napisati detaljni opis dijagrama.}
				\eject
	
		\section{Ostali zahtjevi}
		
%			\textbf{\textit{dio 1. revizije}}\\
%		 
%			 \textit{Nefunkcionalni zahtjevi i zahtjevi domene primjene dopunjuju funkcionalne zahtjeve. Oni opisuju \textbf{kako se sustav treba ponašati} i koja \textbf{ograničenja} treba poštivati (performanse, korisničko iskustvo, pouzdanost, standardi kvalitete, sigurnost...). Primjeri takvih zahtjeva u Vašem projektu mogu biti: podržani jezici korisničkog sučelja, vrijeme odziva, najveći mogući podržani broj korisnika, podržane web/mobilne platforme, razina zaštite (protokoli komunikacije, kriptiranje...)... Svaki takav zahtjev potrebno je navesti u jednoj ili dvije rečenice.}
%			 

		\begin{packed_item}
		
			\item Sustav treba omogućiti rad više korisnika u stvarnom vremenu
			\item 	Korisničko sučelje i sustav moraju podržavati hrvatsku abecedu (dijakirtičke znakove) pri unosu i prikazu tekstualnog sadržaja
			\item Neispravno korištenje korisničkog sučelja ne smije narušiti funkcionalnosti i rad sustava
			\item Sustav treba biti implementiran kao web aplikacija koristeći skriptne jezike
			\item Veza s bazom podataka mora biti zaštićena, brza i otporna na greške.
			\item Sustav treba biti jednostavan za korištenje, korisnici se moraju znati koristiti sučeljem bez opširnih uputa
			\item Nadogradnja sustava ne smije narušavati posotjeće funcionalnosti sustava
			\item Pristup bazi podataka ne smije dugo trajati
			\item Pristup sustavu mora biti omogućen iz javne mreže pomoću HTTPS.
		
		\end{packed_item}
			 
			 
	

	\chapter{Arhitektura i dizajn sustava}
		
%		\textbf{\textit{dio 1. revizije}}\\
%
%		\textit{ Potrebno je opisati stil arhitekture te identificirati: podsustave, preslikavanje na radnu platformu, spremišta podataka, mrežne protokole, globalni upravljački tok i sklopovsko-programske zahtjeve. Po točkama razraditi i popratiti odgovarajućim skicama:}
%	\begin{itemize}
%		\item 	\textit{izbor arhitekture temeljem principa oblikovanja pokazanih na predavanjima (objasniti zašto ste baš odabrali takvu arhitekturu)}
%		\item 	\textit{organizaciju sustava s najviše razine apstrakcije (npr. klijent-poslužitelj, baza podataka, datotečni sustav, grafičko sučelje)}
%		\item 	\textit{organizaciju aplikacije (npr. slojevi frontend i backend, MVC arhitektura) }		
%	\end{itemize}

	Arhitektura sustava koju smo koristili se dijeli na tri podsustava:
	
		\begin{packed_item}
			\item Web poslužitelj
			\item Web aplikacija
			\item Baza podataka
		\end{packed_item}

		\begin{figure}[H]
			\includegraphics[width=\textwidth]{slike/arhitektura.png}
			\centering
			\caption{Arhitektura sustava}
			\label{fig:arhitektura}
		\end{figure}
		
		\textit{\underline{Web preglednik}} je aplikacija koja omogućuje korisniku pristup \textit{World Wide Web}-u. Kada korisnik zatraži web stranicu, preglednik dohvaća potreban sadržaj, web stranicu i potrebne multimedijske sadržaje, s web poslužitelja i prikazuje na korisnikovom uređaju. Web stranica je hipertekstualni dokument koji se prikazuje korisniku na ekranu. Web stranice se sastoje od jedne ili više tekstualnih datoteka pisanih u jezicima: \textit{HTML (HyperText Markup Lagnuage), CSS (Cascade Style Sheets) te JS (JavaScript)}, koje web preglednik prevodi i prikazuje ih korisniku koji putem web preglednika šalje zahtjev web poslužitelju.
		
		\textit{\underline{Web poslužitelj}} je računalni hardveri sofver koji služi za komunikaciju klijenta s aplikacijom. Poslužitelj prima zahtjeve preko \textit{HTPP (Hyper Text Transfer Protocol)} mrežnog protokola stvorenog za distirbuciju web sadržaja. Korisnik preko web preglednika pokreće komunikaciju postavljanjem željenih zahtjeva koristeći \textit{HTTP}, a web poslužitelj, ovisno o zahtjevu koji obrađuje, odgovara sadržajem tog resursa (HTML dokumentom) ili porukom o pogrešci.
		
		Programski jezici koji su korišteni za izradu projekta su PostgreSQL, za baze podataka, te JavaScript (Node.js za backend), te React (za frontend). Odabrano razvojno okruženje je IntelliJ. Arhitektura sustava temelji se na MVC (Model-View-Controller) obrascu koji funkcionira na načelu dizajna \textit{Separation of Concerns (SoC)}. SoC omogućava odvajanje programa u različite odjeljke tako da se svaki odjeljak bavi svojim dijelom, što olakšava izradu i daljnji razvoj aplikacije.
		
		MVC sastoji se od:
		\begin{packed_item}
			\item \textbf{Model} - središnja komponenta sustava koja izravno upravlja podacima, logikom i pravilima aplikacije. Neovisna je o korisničkom sučelju te prima ulazne podatke od Controllera.
			
			\item \textbf{View} - definira kako se podaci aplikacije trebaju prikazivati
			
			\item \textbf{Controller} - zadužen je za primanje i uređivanje zahtjeva koje prosljeđuje na Model ili View. Upravalja korisničkim zahtjevima i na temelju njih izvodi daljnje korake potrebne s ostalim elementima.
		\end{packed_item}
		
		\eject
		

				
		\section{Baza podataka}
			
%			\textbf{\textit{dio 1. revizije}}\\
%			
%		\textit{Potrebno je opisati koju vrstu i implementaciju baze podataka ste odabrali, glavne komponente od kojih se sastoji i slično.}

			Za potrebe sustava našeg projekta korištena je relacijska baza podataka. Osnovna građa baze je relacija, odnosno tablica koja je definirana imenom i skupom atributa. Glavni zadatak baze podataka je brza i jednostavna pohrana, izmjena i dohvat podataka za obradu. Baza podataka ove aplikacije sastoji se od tablica:
			
			\begin{packed_item}
				\item User
				\item Genre
				\item Subgenre
				\item Vinyl
				\item Artist
				\item Ad
				\item Sale ad
				\item Exchange ad
				\item Event
			\end{packed_item}
		
			\subsection{Opis tablica}
			
%
%				\textit{Svaku tablicu je potrebno opisati po zadanom predlošku. Lijevo se nalazi točno ime varijable u bazi podataka, u sredini se nalazi tip podataka, a desno se nalazi opis varijable. Svjetlozelenom bojom označite primarni ključ. Svjetlo plavom označite strani ključ}
				
				\textbf{User}	Ovaj entitet sadrži sve važne informacije o korisniku aplikacije. Sadrži atribute: user\_ id, name, surname, username, password, email, contact\_ email i location. Ovaj entitet je u vezi \textit{Many-to-Many} sa samim sobom preko atributa user\_ id korisnika, \textit{Many-to-Many} sa entitetom Genre preko genre\_ id, \textit{One-to-Many} sa entitetom Ad preko user\_ id, \textit{Many-to-Many} sa Vinyl preko user\_ id te \textit{One-to-Many} sa Vinyl preko vinyl\_ id i \textit{Many-to-Many} sa Artist preko user\_ id.
				
				%User 
				\begin{longtblr}[
					label=none,
					entry=none
					]{
						width = \textwidth,
						colspec={|X[8,l]|X[6, l]|X[20, l]|}, 
						rowhead = 1,
					} %definicija širine tablice, širine stupaca, poravnanje i broja redaka naslova tablice
					\hline \multicolumn{3}{|c|}{\textbf{User}}	 \\ \hline[3pt]
					\SetCell{LightGreen}user\_ id & LONG	&  Jedinstveni identifikator korisnika 	\\ \hline
					name	& STRING & Ime korisnika  	\\ \hline 
					surname & STRING & Prezime korisnika  \\ \hline 
					username & STRING	&  Korisničko ime	\\ \hline 
					password & STRING & Lozinka \\ \hline
					email & STRING & E-mail korisnika \\ \hline
					contact\_ email & STRING & E-mail za stupanje u kontakt s korisnikom \\ \hline
					location & GEOGRAPHY & Geografska lokacija uređaja \\ \hline
				\end{longtblr}
				
				\textbf{Genre}	Ovaj entitet sadrži sve važne informacije o žanrovima. Sadrži atribute: genre\_ id i name. Ovaj entitet je u vezi \textit{Many-to-Many} sa User preko atributa user\_ id, \textit{One-to-Many} sa Subgenre preko genre\_ id i \textit{One-to-Many} sa Vinyl preko genre\_ id.
				
				%Genre
				\begin{longtblr}[
					label=none,
					entry=none
					]{
						width = \textwidth,
						colspec={|X[8,l]|X[6, l]|X[20, l]|}, 
						rowhead = 1,
					} 
					\hline \multicolumn{3}{|c|}{\textbf{Genre}}	 \\ \hline[3pt]
					\SetCell{LightGreen}genre\_ id & LONG	&  Jedinstveni identifikator žanra  \\ \hline
					name	 & STRING & Ime žanra  	\\ \hline 
				\end{longtblr}
				
				\textbf{Subgenre}	Ovaj entitet sadrži sve važne informacije o podžanrovima. Sadrži atribute: subgenre\_ id, name i parent\_ genre\_ id. Ovaj entitet je u vezi \textit{Many-to-One} sa Genre preko atributa genre_id.
				
				%Subgenre
				\begin{longtblr}[
					label=none,
					entry=none
					]{
						width = \textwidth,
						colspec={|X[8,l]|X[6, l]|X[20, l]|}, 
						rowhead = 1,
					} 
					\hline \multicolumn{3}{|c|}{\textbf{Genre}}	 \\ \hline[3pt]
					\SetCell{LightGreen}subgenre\_ id & LONG	&  Jedinstveni identifikator podžanra  \\ \hline
					name	& STRING & Ime podžanra  	\\ \hline
					\SetCell{LightBlue}parent\_ genre\_ id & LONG & Identifikator roditeljskog žanra \\ \hline
					
				\end{longtblr}
				
				\textbf{Vinyl}	Ovaj entitet sadrži sve važne informacije o vinilima. Sadrži atribute: vinyl\_ id, album\_ name, release\_ year, genre\_ id, subgenre\_ id, image, condition\_ evaluation, isRare, description, value\_ kn, RPM, diameter, capacity, reproduction\_ quality, nmb\_ of\_ audio\_ channels, time\_ of\_ reproduction. Ovaj entitet je u vezi \textit{One-to-Many} sa Ad preko atributa vinyl\_ id, \textit{Many-to-Many} sa User preko user\_ id, \textit{Many-to-One} sa User preko vinyl\_ id, \textit{Many-to-One} sa Genre preko genre\_ id i \textit{Many-to-One} sa Artist preko artist\_ id.
				
				%Vinyl
				\begin{longtblr}[
					label=none,
					entry=none
					]{
						width = \textwidth,
						colspec={|X[8,l]|X[6, l]|X[20, l]|}, 
						rowhead = 1,
					} 
					\hline \multicolumn{3}{|c|}{\textbf{Vinyl}}	 \\ \hline[3pt]
					\SetCell{LightGreen}vinyl\_ id & LONG	&  Jedinstveni identifikator vinila  \\ \hline
					album\_ name	& STRING & Ime albuma \\ \hline 
					release\_ year & INT & Godina izdanja \\ \hline
					\SetCell{LightBlue}genre\_ id & LONG & Jedinstveni identifikator žanra \\ \hline
					subgenre\_ id & LONG & Jednistveni identifikator podžanra \\ \hline
					image & BYTEA & Slika vinila \\ \hline
					condition\_ evaluation & INT & Ocjena stanja vinila \\ \hline
					isRare & BOOLEAN & Raritet vinila \\ \hline
					description & STRING & Kratki opis vinila \\ \hline
					value\_ kn & NUMERIC & Vrijednost vinila \\ \hline
					RPM & STRING & Brzina okretaja u minuti \\ \hline
					diameter & FLOAT & Promjer vinila \\ \hline
					capacity & STRING & Kapacitet vinila \\ \hline
					reproduction\_ quality & STRING & Kvaliteta reprodukcije vinila \\ \hline
					nmb\_ of\_ audio\_ channels & INT & Broj audio kanala \\ \hline
					time\_ of\_ reproduction & TIME & Vrijeme reprodukcije \\ \hline
					
				\end{longtblr}
				
				\textbf{Artist}	Ovaj entitet sadrži sve važne informacije o izvođaču. Sadrži atribute: artist\_ id i name. Ovaj entitet je u vezi \textit{Many-to-Many} sa User preko atributa user\_ id i \textit{One-to-Many} sa Vinyl preko artist\_ id.
				
				%Artist
				\begin{longtblr}[
					label=none,
					entry=none
					]{
						width = \textwidth,
						colspec={|X[8,l]|X[6, l]|X[20, l]|}, 
						rowhead = 1,
					} 
					\hline \multicolumn{3}{|c|}{\textbf{Vinyl}}	 \\ \hline[3pt]
					\SetCell{LightGreen}artist\_ id & LONG	&  Jedinstveni identifikator izvođača  \\ \hline
					name & STRING & Ime izvođača \\ \hline
					
				\end{longtblr}
				
				\textbf{Ad}	Ovaj entitet sadrži sve važne informacije o oglasu. Sadrži atribute: ad\_ id, vinyl\_ id, sale\_ ad\_ id, exchange\_ ad\_ id i isActive. Ovaj entitet je u vezi \textit{Many-to-One} sa User preko user\_ id, \textit{Many-to-One} sa Vinyl preko vinyl\_ id, \textit{One-to-One} sa SaleAd preko ad\_ id i \textit{One-to-One} sa ExchangeAd preko ad\_ id.
				
				%Ad
				\begin{longtblr}[
					label=none,
					entry=none
					]{
						width = \textwidth,
						colspec={|X[8,l]|X[6, l]|X[20, l]|}, 
						rowhead = 1,
					} 
					\hline \multicolumn{3}{|c|}{\textbf{Ad}}	 \\ \hline[3pt]
					\SetCell{LightGreen}ad\_ id & LONG	&  Jedinstveni identifikator oglasa  \\ \hline
					vinyl\_ id & STRING & Jedinstveni identifikator vinila \\ \hline
					\SetCell{LightBlue}sale\_ ad\_ id & LONG & Jednistveni identifikator oglasa za prodaju \\ \hline
					\SetCell{LightBlue}exchange\_ ad\_ id & LONG & Jedinstveni identifikator oglasa za razmjenu \\ \hline
					isActive & BOOLEAN & Aktivnost oglasa \\ \hline
					
				\end{longtblr}
				
				\textbf{SaleAd}	Ovaj entitet sadrži sve važe informacije za oglas koji je označen za prodaju vinila. Sadrži atribute: ad\_ id i price. Ovaj entitet je u vezi \textit{One-to-One} sa Ad preko atributa ad\_ id.
				
				%SaleAd
				\begin{longtblr}[
					label=none,
					entry=none
					]{
						width = \textwidth,
						colspec={|X[8,l]|X[6, l]|X[20, l]|}, 
						rowhead = 1,
					} 
					\hline \multicolumn{3}{|c|}{\textbf{SaleAd}}	 \\ \hline[3pt]
					\SetCell{LightGreen}ad\_ id & LONG	&  Jedinstveni identifikator oglasa  \\ \hline
					price & NUMERIC & Cijena vinila u oglasu \\ \hline
					
				\end{longtblr}
				
				\textbf{ExchangeAd}	Ovaj entitet sadrži sve važe informacije za oglas koji je označen za razmjenu vinila. Sadrži atribute: ad\_ id, new\_ Owner\_ id i exchanged\_ vinyl\_ id. Ovaj entitet je u vezi \textit{One-to-One} sa Ad preko atributa ad\_ id.
				
				%ExchangeAd
				\begin{longtblr}[
					label=none,
					entry=none
					]{
						width = \textwidth,
						colspec={|X[8,l]|X[6, l]|X[20, l]|}, 
						rowhead = 1,
					} 
					\hline \multicolumn{3}{|c|}{\textbf{ExchangeAd}}	 \\ \hline[3pt]
					\SetCell{LightGreen}ad\_ id & LONG	&  Jedinstveni identifikator oglasa  \\ \hline
					new\_ Owner\_ id & LONG & Jedinstveni identifikator novog vlasnika \\ \hline
					exchanged\_ vinyl\_ id & LONG & Jedinstveni identifikator razmjenjenog vinila \\ \hline
					
				\end{longtblr}
				
				\textbf{SaleAd}	Ovaj entitet sadrži sve važe informacije za događaj. Sadrži atribute: event\_ id, title, image, creation\_ date, description, social\_ network\_ link i location. Ovaj entitet nije u vezi niti sa jednim drugim entitetom.
				
				%Event
				\begin{longtblr}[
					label=none,
					entry=none
					]{
						width = \textwidth,
						colspec={|X[8,l]|X[6, l]|X[20, l]|}, 
						rowhead = 1,
					} 
					\hline \multicolumn{3}{|c|}{\textbf{Event}}	 \\ \hline[3pt]
					\SetCell{LightGreen}event\_ id & LONG	&  Jedinstveni identifikator događaja  \\ \hline
					title & STRING & Naslov događaja \\ \hline
					image & BYTEA & Slika događaja \\ \hline
					creation\_ date & DATE & Datum događaja \\ \hline
					description & STRING & Kratki opis događaja \\ \hline
					social\_ network\_ link & STRING & Link na događaj \\ \hline
					location & GEOGRAPHY & Lokacija događaja \\ \hline
					
				\end{longtblr}
			
			\subsection{Dijagram baze podataka}
%				\textit{ U ovom potpoglavlju potrebno je umetnuti dijagram baze podataka. Primarni i strani ključevi moraju biti označeni, a tablice povezane. Bazu podataka je potrebno normalizirati. Podsjetite se kolegija "Baze podataka".}

				\begin{figure}[H]
					\includegraphics[width=\textwidth]{slike/baza.jpg}
					\centering
					\caption{Dijagram baze podataka}
					\label{fig:baza}
				\end{figure}
			
			
			\eject
			
			
		\section{Dijagram razreda}
		
%			\textit{Potrebno je priložiti dijagram razreda s pripadajućim opisom. Zbog preglednosti je moguće dijagram razlomiti na više njih, ali moraju biti grupirani prema sličnim razinama apstrakcije i srodnim funkcionalnostima.}\\
%			
%			\textbf{\textit{dio 1. revizije}}\\
%			
%			\textit{Prilikom prve predaje projekta, potrebno je priložiti potpuno razrađen dijagram razreda vezan uz \textbf{generičku funkcionalnost} sustava. Ostale funkcionalnosti trebaju biti idejno razrađene u dijagramu sa sljedećim komponentama: nazivi razreda, nazivi metoda i vrste pristupa metodama (npr. javni, zaštićeni), nazivi atributa razreda, veze i odnosi između razreda.}\\

			

			
%			\textbf{\textit{dio 2. revizije}}\\			
%			
%			\textit{Prilikom druge predaje projekta dijagram razreda i opisi moraju odgovarati stvarnom stanju implementacije}
			
			
			
			\eject
		
%		\section{Dijagram stanja}
%			
%			
%			\textbf{\textit{dio 2. revizije}}\\
%			
%			\textit{Potrebno je priložiti dijagram stanja i opisati ga. Dovoljan je jedan dijagram stanja koji prikazuje \textbf{značajan dio funkcionalnosti} sustava. Na primjer, stanja korisničkog sučelja i tijek korištenja neke ključne funkcionalnosti jesu značajan dio sustava, a registracija i prijava nisu. }
%			
%			
%			\eject 
%		
%		\section{Dijagram aktivnosti}
%			
%			\textbf{\textit{dio 2. revizije}}\\
%			
%			 \textit{Potrebno je priložiti dijagram aktivnosti s pripadajućim opisom. Dijagram aktivnosti treba prikazivati značajan dio sustava.}
%			
%			\eject
%		\section{Dijagram komponenti}
%		
%			\textbf{\textit{dio 2. revizije}}\\
%		
%			 \textit{Potrebno je priložiti dijagram komponenti s pripadajućim opisom. Dijagram komponenti treba prikazivati strukturu cijele aplikacije.}

	%\chapter{Implementacija i korisničko sučelje}
		
		
		\section{Korištene tehnologije i alati}
		
%			\textbf{\textit{dio 2. revizije}}
%			
%			 \textit{Detaljno navesti sve tehnologije i alate koji su primijenjeni pri izradi dokumentacije i aplikacije. Ukratko ih opisati, te navesti njihovo značenje i mjesto primjene. Za svaki navedeni alat i tehnologiju je potrebno \textbf{navesti internet poveznicu} gdje se mogu preuzeti ili više saznati o njima}.
			
			Tehnologije koje se bile korištene u svrhu komunikacije tima su bile WhatsApp i Discord. Za izradu obrazaca uporabe i UML dijagrama korišten je alat Astah UML, a kao sustav za upravljanje kodom projekta Git. Kompletni kod i dokumentacija projekta dostupna je na udaljenom repozitoriju platforme Gitlab.
			
			Kao razvojno okruženje korišteno je IntelliJ IDEA, integrirano razvojno okruženje razvijeno od tvrtke JetBrains. Napisano je u Javi u svrhu razvoja računalnog softvera. IDE pruža integraciju s alatima za izradu/pakiranje kao što su bower, gradle, i sl. Podržava sustave kao Git i SVN te baze podataka tipa Microsoft SQL Server, Oracle, PostrgeSQL, i sl.
			
			\textit{Frontend} dio web aplikacije napisan je korištenjem Javascript biblioteke React i programskog jezika JavaScript, dok je \textit{backend} dio napisan je korištenjem radnog okvira Spring Boot i programskog jezika Java. React (također poznat kao React.js ili ReactJS) open-source biblioteka za izgradnju korisničkih sučelja temeljenih na UI komponentama. React se bavi samo upravljanjem stanjem i prikazivanjem tog stanja DOM-u, s obzirom na to stvaranje aplikacija obično zahtijeva korištenje dodatnih biblioteka za interakciju s API-jem. Radni okvir Spring Boot olakšava stvaranje samostalnih aplikacija koje se temelje na Springu koje se mogu samo "pokrenuti".
			
			Za upravljanje relacijskom bazom podataka korišten je sustav PostgreSQL, a za pokretanje i rad aplikacije korištena je platforma Heroku.
			
			\eject 
		
	
		\section{Ispitivanje programskog rješenja}
			
%			\textbf{\textit{dio 2. revizije}}\\
%			
%			 \textit{U ovom poglavlju je potrebno opisati provedbu ispitivanja implementiranih funkcionalnosti na razini komponenti i na razini cijelog sustava s prikazom odabranih ispitnih slučajeva. Studenti trebaju ispitati temeljnu funkcionalnost i rubne uvjete.}
	
			
			\subsection{Ispitivanje komponenti}
%			\textit{Potrebno je provesti ispitivanje jedinica (engl. unit testing) nad razredima koji implementiraju temeljne funkcionalnosti. Razraditi \textbf{minimalno 6 ispitnih slučajeva} u kojima će se ispitati redovni slučajevi, rubni uvjeti te izazivanje pogreške (engl. exception throwing). Poželjno je stvoriti i ispitni slučaj koji koristi funkcionalnosti koje nisu implementirane. Potrebno je priložiti izvorni kôd svih ispitnih slučajeva te prikaz rezultata izvođenja ispita u razvojnom okruženju (prolaz/pad ispita). }
			
			
			
			\subsection{Ispitivanje sustava}
			
%			 \textit{Potrebno je provesti i opisati ispitivanje sustava koristeći radni okvir Selenium\footnote{\url{https://www.seleniumhq.org/}}. Razraditi \textbf{minimalno 4 ispitna slučaja} u kojima će se ispitati redovni slučajevi, rubni uvjeti te poziv funkcionalnosti koja nije implementirana/izaziva pogrešku kako bi se vidjelo na koji način sustav reagira kada nešto nije u potpunosti ostvareno. Ispitni slučaj se treba sastojati od ulaza (npr. korisničko ime i lozinka), očekivanog izlaza ili rezultata, koraka ispitivanja i dobivenog izlaza ili rezultata.\\ }
%			 
%			 \textit{Izradu ispitnih slučajeva pomoću radnog okvira Selenium moguće je provesti pomoću jednog od sljedeća dva alata:}
%			 \begin{itemize}
%			 	\item \textit{dodatak za preglednik \textbf{Selenium IDE} - snimanje korisnikovih akcija radi automatskog ponavljanja ispita	}
%			 	\item \textit{\textbf{Selenium WebDriver} - podrška za pisanje ispita u jezicima Java, C\#, PHP koristeći posebno programsko sučelje.}
%			 \end{itemize}
%		 	\textit{Detalji o korištenju alata Selenium bit će prikazani na posebnom predavanju tijekom semestra.}
%			
			\eject 
		
		
		\section{Dijagram razmještaja}
			
%			\textbf{\textit{dio 2. revizije}}
%			
%			 \textit{Potrebno je umetnuti \textbf{specifikacijski} dijagram razmještaja i opisati ga. Moguće je umjesto specifikacijskog dijagrama razmještaja umetnuti dijagram razmještaja instanci, pod uvjetom da taj dijagram bolje opisuje neki važniji dio sustava.}

		Dijagram razmještaja je strukturni dijagram koji opisuje topologiju sustava i usredotočen je na odnos sklopovskih i programskih dijelova. Sustav je baziran na trorazinskoj arhitekturi pri kojoj klijent - poslužitelj i poslužitelj - baza podataka komuniciraju preko HTTP veze. Klijenti upotrebljavaju web preglednik kako bi pristupili web aplikaciji, dok web aplikacija pristupa bazi podataka kako bi pristupala podacima u svrhu provjere ili prikaza korisniku.

				\begin{figure}[H]
					\includegraphics[width=\textwidth]{slike/DD_web_aplikacija.png}
					\centering
					\caption{Dijagram razmještaja}
					\label{fig:dig_razmjestaja}
				\end{figure}
			
			\eject 
		
		\section{Upute za puštanje u pogon}
		
%			\textbf{\textit{dio 2. revizije}}\\
%		
%			 \textit{U ovom poglavlju potrebno je dati upute za puštanje u pogon (engl. deployment) ostvarene aplikacije. Na primjer, za web aplikacije, opisati postupak kojim se od izvornog kôda dolazi do potpuno postavljene baze podataka i poslužitelja koji odgovara na upite korisnika. Za mobilnu aplikaciju, postupak kojim se aplikacija izgradi, te postavi na neku od trgovina. Za stolnu (engl. desktop) aplikaciju, postupak kojim se aplikacija instalira na računalo. Ukoliko mobilne i stolne aplikacije komuniciraju s poslužiteljem i/ili bazom podataka, opisati i postupak njihovog postavljanja. Pri izradi uputa preporučuje se \textbf{naglasiti korake instalacije uporabom natuknica} te koristiti što je više moguće \textbf{slike ekrana} (engl. screenshots) kako bi upute bile jasne i jednostavne za slijediti.}
%			
%			
%			 \textit{Dovršenu aplikaciju potrebno je pokrenuti na javno dostupnom poslužitelju. Studentima se preporuča korištenje neke od sljedećih besplatnih usluga: \href{https://aws.amazon.com/}{Amazon AWS}, \href{https://azure.microsoft.com/en-us/}{Microsoft Azure} ili \href{https://www.heroku.com/}{Heroku}. Mobilne aplikacije trebaju biti objavljene na F-Droid, Google Play ili Amazon App trgovini.}
			
			
			\eject 
	%\chapter{Zaključak i budući rad}
		
%		\textbf{\textit{dio 2. revizije}}\\
%		
%		 \textit{U ovom poglavlju potrebno je napisati osvrt na vrijeme izrade projektnog zadatka, koji su tehnički izazovi prepoznati, jesu li riješeni ili kako bi mogli biti riješeni, koja su znanja stečena pri izradi projekta, koja bi znanja bila posebno potrebna za brže i kvalitetnije ostvarenje projekta i koje bi bile perspektive za nastavak rada u projektnoj grupi.}
%		
%		 \textit{Potrebno je točno popisati funkcionalnosti koje nisu implementirane u ostvarenoj aplikaciji.}

		Projektni zadatak naše grupe bio je izrada i razvoj web aplikacije za stvaranje kolekcije vinila uz mogućnost razmjene, kupovine i prodaje vinila. Zadatak je ostvaren nakon 16 tjedana grupnog rada tima. Provedba praćenja napretka razvoja aplikacije tima bila je kroz dvije faze.
		
		Prva faza projekta krenula je sporije zbog upoznavanja tima i njihovih iskustva i znanja, organizacije i raspodijele poslova te dogovoru za sastanke koji će pratiti napredak projekta. Uspješnom raspodijelom tima na podtimove obilježio je početak rada na projetku. Cilj prve faze projekta bila je izrada \textit{alfa} inačice projekta, tj. web aplikacija s osnovim radnim funkcionalnostima kao što su \textit{login} i \textit{registracija}. Velika pomoć u izradi i razvoju \textit{frontend-a} i \textit{backend-a} bili su obrasci i dijagrami (obrasci uporabe, sekvencijski dijagrami, dijgram razreda i model baze podataka). Vizualni dizajn web aplikacije bio je brzo riješen kako bi tim mogao nastaviti dalje implementirati funkcije.
		
		Druga faza projekta, iako kraća od prve, ali zahtijevnija, krenula je intenzivnije po pitanju posla koji je trebao svaki član tima odraditi. Prilikom izrade implementacijskih rješenja članovi tima morali su samostalno naučiti odabrane alate i programske jezike kako bi zadovoljili dogovoreni cilj. Također je bilo potrebno napraviti korektne UML dijagrame i ispravnu dokumentaciju kako bi budući korisnici, investitori mogli lakše razumijeti funckionalnost sustava. Kvalitetno izrađen prototip aplikacije i složnost tima, pomogao nam je u drugoj fazi projekta kako bi se mogle izbjeći vremensko složne pogreške i nepotrebne funckionalnosti aplikacije. Slijedeći korak projektnog zadatka bio bi izrada mobline aplikacije koja bi mogla otvoriti nove funckionalnosti aplikacije (npr. prepoznavanje ploče s kamerom koja bi popunila osnovne informacije ploče, ...).
		
		Najvažniji dio projekta bila je organizacija i komunikacija tima. Prateći napredak i ostatak posla svakog podtima bila je nužna kako bi se projektni zadatak mogao napraviti u roku. Svaki član tima sudjelovanjem u ovom projektu shvatio je veliku važnost u dobroj vremenskoj organiziaciji i komunikaciji i koordinaciji s ostatkom tima. Zadovoljni smo s količinom posla koje smo odradili u zadanom vremenskom periodu i postoji još mjesta za poboljšanje, ali to je sve posljedica neiskustva, koje svi u početku imamo, na takvim i sličnim budućim projektima. 
		
		\eject 
	\chapter*{Popis literature}
		\addcontentsline{toc}{chapter}{Popis literature}
	 	
% 		\textbf{\textit{Kontinuirano osvježavanje}}
%	
%		\textit{Popisati sve reference i literaturu koja je pomogla pri ostvarivanju projekta.}
%		
		
		\begin{enumerate}
			
			
			\item  Programsko inženjerstvo, FER ZEMRIS, \url{http://www.fer.hr/predmet/proinz}
			
			\item  I. Sommerville, "Software engineering", 8th ed, Addison Wesley, 2007.
			
			\item  T.C.Lethbridge, R.Langaniere, "Object-Oriented Software Engineering", 2nd ed. McGraw-Hill, 2005.
			
			\item  I. Marsic, Software engineering book``, Department of Electrical and Computer Engineering, Rutgers University, \url{http://www.ece.rutgers.edu/~marsic/books/SE}
			
			\item  The Unified Modeling Language, \url{https://www.uml-diagrams.org/}
			
			\item  Astah Community, \url{http://astah.net/editions/uml-new}
			
			\item Njuškalo, \url{https://www.njuskalo.hr/}
			
			\item Oglasnik, \url{https://www.oglasnik.hr/}
		\end{enumerate}
		
		 
	
	
	\begingroup
	\renewcommand*\listfigurename{Indeks slika i dijagrama}
	%\renewcommand*\listtablename{Indeks tablica}
	%\let\clearpage\relax
	\listoffigures
	%\vspace{10mm}
	%\listoftables
	\endgroup
	\addcontentsline{toc}{chapter}{Indeks slika i dijagrama}


	
	\eject 
		
	\chapter*{Dodatak: Prikaz aktivnosti grupe}
		\addcontentsline{toc}{chapter}{Dodatak: Prikaz aktivnosti grupe}
		
		\section*{Dnevnik sastajanja}
		
%		\textbf{\textit{Kontinuirano osvježavanje}}\\
%		
%		 \textit{U ovom dijelu potrebno je redovito osvježavati dnevnik sastajanja prema predlošku.}
		
		\begin{packed_enum}
			\item  sastanak
			
			\item[] \begin{packed_item}
				\item Datum: 7. listopada 2021.
				\item Prisustvovali: J.Srzić, P.Pavlić, T.Krišto, L.Aleksić, M.Dulibić, P.Cukrov, T.Drezga
				\item Teme sastanka:
				\begin{packed_item}
					\item  upoznavanje s timom
					\item  podijela uloga u timu
				\end{packed_item}
			\end{packed_item}
			
			\item  sastanak
			\item[] \begin{packed_item}
				\item Datum: 14. listopada 2021.
				\item Prisustvovali: J.Srzić, P.Pavlić, T.Krišto, L.Aleksić, M.Dulibić, P.Cukrov, T.Drezga
				\item Teme sastanka:
				\begin{packed_item}
					\item  rasprava o temi projekta
					\item  nejasnoće oko teksta zadatka projekta
				\end{packed_item}
			\end{packed_item}
			
			\item  sastanak
			\item[] \begin{packed_item}
				\item Datum: 21. listopada 2021.
				\item Prisustvovali: J.Srzić, P.Pavlić, T.Krišto, L.Aleksić, M.Dulibić, P.Cukrov, T.Drezga
				\item Teme sastanka:
				\begin{packed_item}
					\item  razjašnjavanje nejasnoća s asistentima
					\item  predlaganje nekih ideja i alternativih rješenja za pojedine dijelove zadataka
				\end{packed_item}
			\end{packed_item}
			
			\item  sastanak
			\item[] \begin{packed_item}
				\item Datum: 26. listopada 2021.
				\item Prisustvovali: J.Srzić, P.Pavlić, T.Krišto, L.Aleksić, M.Dulibić, P.Cukrov, T.Drezga
				\item Teme sastanka:
				\begin{packed_item}
					\item  upoznavanje sa Gitom
					\item  dogovor oko razvojnog okruženja
				\end{packed_item}
			\end{packed_item}
			
			\item  sastanak
			\item[] \begin{packed_item}
				\item Datum: 4. studenoga 2021.
				\item Prisustvovali: T.Krišto, L.Aleksić, M.Dulibić, P.Cukrov, T.Drezga
				\item Teme sastanka:
				\begin{packed_item}
					\item  razrada napravljene baze podataka
					\item  određivanje funkcionalnosti prve inačice projekta
				\end{packed_item}
			\end{packed_item}
			
			\item  sastanak
			\item[] \begin{packed_item}
				\item Datum: 7. studenoga 2021.
				\item Prisustvovali: J.Srzić, P.Pavlić, T.Krišto, L.Aleksić, M.Dulibić, P.Cukrov, T.Drezga
				\item Teme sastanka:
				\begin{packed_item}
					\item  tjedni nalazak tko je što napravio te kako koji dio projekta napreduje
					\item  što treba biti gotovo do slijedećeg nalaska 
				\end{packed_item}
			\end{packed_item}
			
			\item  sastanak
			\item[] \begin{packed_item}
				\item Datum: 13. studenoga 2021.
				\item Prisustvovali: J.Srzić, P.Pavlić, T.Krišto, L.Aleksić, M.Dulibić, P.Cukrov
				\item Teme sastanka:
				\begin{packed_item}
					\item  backend update
					\item  frontend update
				\end{packed_item}
			\end{packed_item}
			
			%
			
		\end{packed_enum}
		
		\eject
		\section*{Tablica aktivnosti}
		
%			\textbf{\textit{Kontinuirano osvježavanje}}\\
%			
%			 \textit{Napomena: Doprinose u aktivnostima treba navesti u satima po članovima grupe po aktivnosti.}

			\begin{longtblr}[
					label=none,
				]{
					vlines,hlines,
					width = \textwidth,
					colspec={X[7, l]X[1, c]X[1, c]X[1, c]X[1, c]X[1, c]X[1, c]X[1, c]}, 
					vline{1} = {1}{text=\clap{}},
					hline{1} = {1}{text=\clap{}},
					rowhead = 1,
				} 
				\multicolumn{1}{c|}{} & \multicolumn{1}{c|}{\rotatebox{90}{\textbf{Josip Srzić }}} & \multicolumn{1}{c|}{\rotatebox{90}{\textbf{Pino Pavlić }}} &	\multicolumn{1}{c|}{\rotatebox{90}{\textbf{Tea Krišto }}} & \multicolumn{1}{c|}{\rotatebox{90}{\textbf{Lucija Aleksić }}} &	\multicolumn{1}{c|}{\rotatebox{90}{\textbf{Marta Dulibić }}} & \multicolumn{1}{c|}{\rotatebox{90}{\textbf{Petra Cukrov }}} &	\multicolumn{1}{c|}{\rotatebox{90}{\textbf{Toni Drezga }}} \\  
				Upravljanje projektom 		&  &  &  &  &  &  & \\ 
				Opis projektnog zadatka 	&  &  &  &  &  &  & \\ 
				
				Funkcionalni zahtjevi       &  &  &  &  &  &  &  \\ 
				Opis pojedinih obrazaca 	&  &  &  &  &  &  &  \\ 
				Dijagram obrazaca 			&  &  &  &  &  &  &  \\ 
				Sekvencijski dijagrami 		&  &  &  &  &  &  &  \\ 
				Opis ostalih zahtjeva 		&  &  &  &  &  &  &  \\ 

				Arhitektura i dizajn sustava	 &  &  &  &  &  &  &  \\ 
				Baza podataka				&  &  &  &  &  &  &   \\ 
				Dijagram razreda 			&  &  &  &  &  &  &   \\ 
				Dijagram stanja				&  &  &  &  &  &  &  \\ 
				Dijagram aktivnosti 		&  &  &  &  &  &  &  \\ 
				Dijagram komponenti			&  &  &  &  &  &  &  \\ 
				Korištene tehnologije i alati 		&  &  &  &  &  &  &  \\ 
				Ispitivanje programskog rješenja 	&  &  &  &  &  &  &  \\ 
				Dijagram razmještaja			&  &  &  &  &  &  &  \\ 
				Upute za puštanje u pogon 		&  &  &  &  &  &  &  \\  
				Dnevnik sastajanja 			&  &  &  &  &  &  &  \\ 
				Zaključak i budući rad 		&  &  &  &  &  &  &  \\  
				Popis literature 			&  &  &  &  &  &  &  \\  
				&  &  &  &  &  &  &  \\ \hline 
				\textit{Dodatne stavke kako ste podijelili izradu aplikacije} 			&  &  &  &  &  &  &  \\ 
				\textit{npr. izrada početne stranice} 				&  &  &  &  &  &  &  \\  
				\textit{izrada baze podataka} 		 			&  &  &  &  &  &  & \\  
				\textit{spajanje s bazom podataka} 							&  &  &  &  &  &  &  \\ 
				\textit{back end} 							&  &  &  &  &  &  &  \\  
				 							&  &  &  &  &  &  &\\ 
			\end{longtblr}
					
					
		\eject
%		\section*{Dijagrami pregleda promjena}
%		
%		\textbf{\textit{dio 2. revizije}}\\
%		
%		\textit{Prenijeti dijagram pregleda promjena nad datotekama projekta. Potrebno je na kraju projekta generirane grafove s gitlaba prenijeti u ovo poglavlje dokumentacije. Dijagrami za vlastiti projekt se mogu preuzeti s gitlab.com stranice, u izborniku Repository, pritiskom na stavku Contributors.}
		
	


\end{document} %naredbe i tekst nakon ove naredbe ne ulaze u izgrađen dokument 


